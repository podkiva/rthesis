\begin{frame}[noframenumbering,plain]
    \setcounter{framenumber}{1}
    \maketitle
\end{frame}

\begin{frame}
	\frametitle{Цель и задачи}
	
	\textbf{Цель:} повышение эффективности инфокоммуникационных систем путём разработки новых методов оценивания параметров каналов связи для широкополосных каналов связи, каналов связи с отражателями в ближнем геометрическом поле, а также для каналов связи с негауссовым распределением аддитивных помех.
	
	\textbf{Задачи}:
	\begin{enumerate}
		\item Исследование и систематизация параметрические модели каналов связи.
		\item Разработка методики оценивания параметров широкополосных каналов связи.
		\item Разработка алгоритма оценивания параметров канала связи с отражателями в ближнем геометрическом поле с использованием сферической модели фронта волны.
		\item Анализ потенциальных характеристик оценивания параметров каналов связи с негауссовым распределением аддитивной помехи с помощью вычисления границы Крамера-Рао.
	\end{enumerate}

	%% authorvak
	\nocite{vakbib1}%
	\nocite{vakbib2}%
	%
	%% authorwos
	%
	%% authorscopus
	\nocite{Zhang2017}%
	\nocite{Podkurkov2017}%
	\nocite{Podkurkov2019}
	%
	%% authorconf
	\nocite{Podkurkov2018}%
	%% authorother
	\nocite{otherbib1}%
\end{frame}
\note{
	Проговариваются вслух цель и задачи
}

\begin{frame}
    \frametitle{Содержание}
    \tableofcontents
\end{frame}
\note{
    Работа состоит из четырёх глав.

    \medskip
    В первой главе \dots

    Во второй главе \dots

    Третья глава посвящена \dots

    В четвёртой главе \dots
}
