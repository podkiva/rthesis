\chapter*{Введение}                         % Заголовок
\addcontentsline{toc}{chapter}{Введение}    % Добавляем его в оглавление

\newcommand{\actuality}{}
\newcommand{\progress}{}
\newcommand{\aim}{{\textbf\aimTXT}}
\newcommand{\tasks}{\textbf{\tasksTXT}}
\newcommand{\novelty}{\textbf{\noveltyTXT}}
\newcommand{\influence}{\textbf{\influenceTXT}}
\newcommand{\methods}{\textbf{\methodsTXT}}
\newcommand{\defpositions}{\textbf{\defpositionsTXT}}
\newcommand{\reliability}{\textbf{\reliabilityTXT}}
\newcommand{\probation}{\textbf{\probationTXT}}
\newcommand{\contribution}{\textbf{\contributionTXT}}
\newcommand{\publications}{\textbf{\publicationsTXT}}
\newcommand{\passconf}{\underline{\textbf{\passconfTXT}}}


{\actuality} Развитие новых методов модуляции с множеством поднесущих и систем передачи данных с множеством входных и выходных портов (MIMO) увеличивают количество получаемой информации и приводят к тому, что сигналы в таких системах могут быть описаны как многомерные массивы данных - тензоры, представляющие собой дискретизацию многомерных непрерывных сигналов.

Тензорное представление сигналов и моделей каналов связи открывает новые возможности к оцениванию параметров таких сигналов с помощью тензорных разложений. Такие тензорные разложения обладают свойствами уникальности и идентифицируемости, которые необходимы для корректного извлечения полезной информации из принимаемых сигналов и, помимо этого, могут сами нести полезную информацию в определённых сценариях.

Вместе с тем, увеличение объёма получаемой информации приводит к увеличению требований к вычислительным ресурсам таких систем. Моделирование передаточных функций каналов связи как полностью случайных и неизвестных стохастических величин приводит к нереализуемым методам и алгоритмам. Поэтому, в данной работе рассматриваются параметрические модели каналов связи \cite{Richter2005}, позволяющие уменьшить количество свободных параметров каналов связи тем самым снизив требования к вычислительным ресурсам систем связи, а также увеличив эффективность их эквализации.

Новые инфокоммуникационные системы, в погоне за растущими потребностями в пропускной способности, постоянно наращивают используемые ресурсы, чаще всего за счёт использования большей полосы частот. Увеличение относительных полос частот этих систем приводит к несостоятельности традиционных моделей данных в них, и, как следствие, алгоритмов оценивания параметров \cite{DoHong2004, Raimondi2016}. В данной работе предлагается общая методика обработки получаемых данных, позволяющая применять узкополосные алгоритмы оценивания параметров каналов связи - направлений прихода сигналов на массив антенн - в широкополосных системах.

Сферическая модель фронта волны даёт возможность оценивать с помощью массива антенн не только направления прихода сигналов, но и расстояния до источника сигнала или его последнего отражения \cite{Singh2017a, Singh2017}. Получение с помощью тензорных разложений несмещенных оценок фазовых сигнатур приходящих на массив антенн сигналов позволяет применять новые методы определения местоположения источников сигналов и их отражателей в ближнем геометрическом поле \cite{Singh2016}. В данной работе предлагается новый алгоритм оценивания направлений прихода сигналов и расстояния до его источника в ближнем геометрическом поле.

Растущая насыщенность частотного спектра приводит к усложнению внутренних и внешних помех в современных инфокоммуникационных системах, что, в свою очередь, приводит к несостоятельности простых статистических моделей аддитивных помех в виде часто используемого белого шума с нормальным распределением \cite{Kozick2000, Kalyani2012}. Это оправдывает использование более сложных статистических моделей аддитивных помех в таких системах, например, таких как смесей нормальных распределений. Введение более сложных моделей аддитивных помех приводит к необходимости анализа потенциальных характеристик оценивания параметров сигналов, в качестве которых распространено использование нижней границы Крамера-Рао \cite{Cramer1993, KayS.1993}. В данной работе предлагается обобщенный метод расчёта потенциальных характеристик оценивания параметров каналов связи с учётом негауссовских распределений аддитивных помех.

{\object} беспроводные инфокоммуникационные системы с множеством приёмных и/или передающих антенн.

{\predmet} алгоритмы оценивания параметров многомерных сигналов в беспроводных инфокоммуникационных системах.

\begin{comment}
 Обзор, введение в тему, обозначение места данной работы в
мировых исследованиях и~т.\:п., можно использовать ссылки на~другие
работы~\autocite{Gosele1999161}
(если их~нет, то~в~автореферате
автоматически пропадёт раздел <<Список литературы>>). Внимание! Ссылки
на~другие работы в~разделе общей характеристики работы можно
использовать только при использовании \verb!biblatex! (из-за технических
ограничений \verb!bibtex8!. Это связано с тем, что одна
и~та~же~характеристика используются и~в~тексте диссертации, и в
автореферате. В~последнем, согласно ГОСТ, должен присутствовать список
работ автора по~теме диссертации, а~\verb!bibtex8! не~умеет выводить в~одном
файле два списка литературы).
При использовании \verb!biblatex! возможно использование исключительно
в~автореферате подстрочных ссылок
для других работ командой \verb!\autocite!, а~также цитирование
собственных работ командой \verb!\cite!. Для этого в~файле
\verb!common/setup.tex! необходимо присвоить положительное значение
счётчику \verb!\setcounter{usefootcite}{1}!.

Для генерации содержимого титульного листа автореферата, диссертации
и~презентации используются данные из файла \verb!common/data.tex!. Если,
например, вы меняете название диссертации, то оно автоматически
появится в~итоговых файлах после очередного запуска \LaTeX. Согласно
ГОСТ 7.0.11-2011 <<5.1.1 Титульный лист является первой страницей
диссертации, служит источником информации, необходимой для обработки и
поиска документа>>. Наличие логотипа организации на~титульном листе
упрощает обработку и~поиск, для этого разметите логотип вашей
организации в папке images в~формате PDF (лучше найти его в векторном
варианте, чтобы он хорошо смотрелся при печати) под именем
\verb!logo.pdf!. Настроить размер изображения с логотипом можно
в~соответствующих местах файлов \verb!title.tex!  отдельно для
диссертации и автореферата. Если вам логотип не~нужен, то просто
удалите файл с~логотипом.

\ifsynopsis
Этот абзац появляется только в~автореферате.
Для формирования блоков, которые будут обрабатываться только в~автореферате,
заведена проверка условия \verb!\!\verb!ifsynopsis!.
Значение условия задаётся в~основном файле документа (\verb!synopsis.tex! для
автореферата).
\else
Этот абзац появляется только в~диссертации.
Через проверку условия \verb!\!\verb!ifsynopsis!, задаваемого в~основном файле
документа (\verb!dissertation.tex! для диссертации), можно сделать новую
команду, обеспечивающую появление цитаты в~диссертации, но~не~в~автореферате.
\fi

При использовании пакета \verb!biblatex! будут подсчитаны все работы, добавленные
в файл \verb!biblio/author.bib!. Для правильного подсчёта работ в~различных
системах цитирования требуется использовать поля:
\begin{itemize}
\item \texttt{authorvak} если публикация индексирована ВАК,
\item \texttt{authorscopus} если публикация индексирована Scopus,
\item \texttt{authorwos} если публикация индексирована Web of Science,
\item \texttt{authorconf} для докладов конференций,
\item \texttt{authorother} для других публикаций.
\end{itemize}
Для подсчёта используются счётчики:
\begin{itemize}
\item \texttt{citeauthorvak} для работ, индексируемых ВАК,
\item \texttt{citeauthorscopus} для работ, индексируемых Scopus,
\item \texttt{citeauthorwos} для работ, индексируемых Web of Science,
\item \texttt{citeauthorvakscopuswos} для работ, индексируемых одной из трёх баз,
\item \texttt{citeauthorscopuswos} для работ, индексируемых Scopus или Web of~Science,
\item \texttt{citeauthorconf} для докладов на конференциях,
\item \texttt{citeauthorother} для остальных работ,
\item \texttt{citeauthor} для суммарного количества работ.
\end{itemize}
% Счётчик \texttt{citeexternal} используется для подсчёта процитированных публикаций.

Для добавления в список публикаций автора работ, которые не были процитированы в
автореферате требуется их~перечислить с использованием команды \verb!\nocite! в
\verb!Synopsis/content.tex!.

\end{comment}

{\progress} В работе Richter A. \cite{Richter2005}, в которой даётся крайне общее описание параметрической модели радиочастотного канала связи, учитывающей все основные физические эффекты, влияющие на распространение электро-магнитных волн. Однако в данной работе не развивается представление передаточной функции канала связи как многомерного массива данных - тензора, которое ведёт к новым подходам, методам и алгоритмам обработки и оценивания параметров каналов связи.

Описание моделей широкополосных каналов связи и соответствующих алгоритмов оценивания их параметров можно найти в таких работах как \cite{DDW93, HK90, OK90, VB88, WK85, KS90, FW93, CM89, KV96, AeroSK94, AeroCC93}.

Работы по локализации источников сигналов в ближнем геометрическом поле можно найти в \cite{Singh2016, Singh2017a, Singh2017}.

Исследования и алгоритмы оценивания для систем с негауссовским распределением аддитивных помех можно найти в \cite{Kozick2000, Kalyani2012}.

{\aim} данной работы является повышение эффективности инфокоммуникационных систем путём разработки новых алгоритмов оценивания параметров каналов связи, в том числе для широкополосных каналов связи, каналов связи с отражателями в ближнем геометрическом поле, а также для каналов связи с негауссовым распределением аддитивных помех.

Для~достижения поставленной цели необходимо было решить следующие {\tasks}:
\begin{enumerate}
  \item Исследовать и систематизировать параметрические модели каналов связи, в том числе модели широкополосных каналов связи, модели каналов связи с отражателями в ближнем геометрическом поле и модели каналов связи с негауссовым распределением аддитивных помех.
  \item Разработать методику обработки принятых сигналов для широкополосных каналов связи.
  \item Разработать алгоритм оценивания параметров канала связи с отражателями в ближнем геометрическом поле с использованием сферической модели фронта волны.
  \item Разработать метод вычисления границы Крамера-Рао оценивания параметров каналов связи с негауссовым распределением аддитивной помехи.
\end{enumerate}


{\novelty}
\begin{enumerate}
  \item Впервые предложен метод предварительной обработки многомерных сигналов в широкополосных системах, позволяющий применять алгоритмы оценивания параметров каналов связи, разработанные для узкополосных систем.
  \item Разработан новый алгоритм оценивания параметров канала связи с отражателями в ближнем геометрическом поле с использованием сферической модели фронта волны.
  \item Впервые исследована граница Крамера-Рао для задач оценки параметров каналов связи с негауссовым распределением аддитивной помехи, заданным смесью нормальных распределений с ненулевыми средними значениями компонент.
\end{enumerate}

{\influence} Теоретическая значимость работы состоит в следующем:
\begin{itemize}
	\item доказана эффективность метода предварительной обработки многомерных сигналов в широкополосных системах с относительной полосой частот, превышающей 10\%;
	\item показано, что разработанный алгоритм оценивания параметров канала связи с отражателями в ближнем геометрическом поле более эффективен чем существующие алгоритмы;
	\item показано, что использование более сложных моделей аддитивных помех потенциально позволяет увеличить точность работы алгоритмов оценивания параметров каналов связи.
\end{itemize}

Практическая значимость работы состоит в следующем:
\begin{itemize}
	\item программная реализация предложенных алгоритмов;
	\item разработаны компьютерные модели, имитирующие работу инфокоммуникационных систем с предложенными алгоритмами и оценивающие эффективность их работы.	
\end{itemize}

{\methods} При решении поставленных задач научного исследования использовались алгоритмы тензорных разложений, методы линейной алгебры, теория оценивания, теория вероятностей и статистики, методы обработки цифровых сигналов, методы компьютерного моделирования (в частности, метод Монте-Карло) и экспериментального исследования. \ldots

{\defpositions}
\begin{enumerate}
  \item Предложенный метод предварительной обработки многомерных сигналов в широкополосных системах существенно повышает точность алгоритмов оценки параметров каналов связи, разработанных для узкополосных систем. Улучшение точности оценивания увеличивается при увеличении относительной полосы частот рассматриваемой широкополосной системы.
  \item Разработанные алгоритм оценивания параметров канала связи с отражателями в ближнем геометрическом поле обеспечивает лучшую точность оценивания в сравнение с существующими методами.
  \item Исследование границы Крамера-Рао для систем с негауссовским распределением аддитивной помехи, выраженным смесью нормальных распределений, показало потенциальный задел на увеличение эффективности алгоритмов оценивания параметров каналов связи в таких системах.
\end{enumerate}

{\reliability} Достоверность результатов, полученных в ходе данной работы, подтверждается соответствием результатов теоретического анализа  результатам имитационного моделирования, а также результатам других авторов.

{\probation}

{\contribution} Все результаты, приведённые в основных положениях, выносимых на защиту, получены автором самостоятельно. Из работ, опубликованных в соавторстве, в диссертацию включена та их часть, которая получена автором лично.

\begin{comment}
{\passconf} Содержание диссертации соответствует паспорту научной специальности \thesisSpecialtyTwoNumber – \thesisSpecialtyTwoTitle, по пунктам:
\begin{description}
	\item[П.2] Исследование процессов генерации, представления, передачи, хранения и отображения аналоговой, цифровой, видео-, аудио- и мультимедиа информации; разработка рекомендаций по совершенствованию и созданию новых соответствующих алгоритмов и процедур;
	\item[П.8] Исследование и разработка новых сигналов, модемов, кодеков, мультиплексоров и селекторов, обеспечивающих высокую надёжность обмена информацией в условиях воздействия внешних и внутренних помех;
	\item[П.11] Разработка научно-технических основ технологии создания сетей, систем и устройств телекоммуникаций и обеспечения их эффективного функционирования.
\end{description}
\end{comment}

\ifnumequal{\value{bibliosel}}{0}
{%%% Встроенная реализация с загрузкой файла через движок bibtex8. (При желании, внутри можно использовать обычные ссылки, наподобие `\cite{vakbib1,vakbib2}`).
    {\publications} Основные результаты по теме диссертации изложены
    в~XX~печатных изданиях,
    X из которых изданы в журналах, рекомендованных ВАК,
    X "--- в тезисах докладов.
}%
{%%% Реализация пакетом biblatex через движок biber
    \begin{refsection}[bl-author]
        % Это refsection=1.
        % Процитированные здесь работы:
        %  * подсчитываются, для автоматического составления фразы "Основные результаты ..."
        %  * попадают в авторскую библиографию, при usefootcite==0 и стиле `\insertbiblioauthor` или `\insertbiblioauthorgrouped`
        %  * нумеруются там в зависимости от порядка команд `\printbibliography` в этом разделе.
        %  * при использовании `\insertbiblioauthorgrouped`, порядок команд `\printbibliography` в нём должен быть тем же (см. biblio/biblatex.tex)
        %
        % Невидимый библиографический список для подсчёта количества публикаций:
        \printbibliography[heading=nobibheading, section=1, env=countauthorvak,          keyword=biblioauthorvak]%
        \printbibliography[heading=nobibheading, section=1, env=countauthorwos,          keyword=biblioauthorwos]%
        \printbibliography[heading=nobibheading, section=1, env=countauthorscopus,       keyword=biblioauthorscopus]%
        \printbibliography[heading=nobibheading, section=1, env=countauthorconf,         keyword=biblioauthorconf]%
        \printbibliography[heading=nobibheading, section=1, env=countauthorother,        keyword=biblioauthorother]%
        \printbibliography[heading=nobibheading, section=1, env=countauthor,             keyword=biblioauthor]%
        \printbibliography[heading=nobibheading, section=1, env=countauthorvakscopuswos, filter=vakscopuswos]%
        \printbibliography[heading=nobibheading, section=1, env=countauthorscopuswos,    filter=scopuswos]%
        %
        \nocite{*}%
        %
        {\publications} Основные результаты по теме диссертации изложены в~\arabic{citeauthor}~печатных изданиях,
        \arabic{citeauthorvak} из которых изданы в журналах, рекомендованных ВАК\sloppy%
        \ifnum \value{citeauthorscopuswos}>0%
            , \arabic{citeauthorscopuswos} "--- в~периодических научных журналах, индексируемых Web of~Science и Scopus\sloppy%
        \fi%
        \ifnum \value{citeauthorconf}>0%
            , \arabic{citeauthorconf} "--- в~тезисах докладов.
        \else%
            .
        \fi
    \end{refsection}%
    \begin{refsection}[bl-author]
        % Это refsection=2.
        % Процитированные здесь работы:
        %  * попадают в авторскую библиографию, при usefootcite==0 и стиле `\insertbiblioauthorimportant`.
        %  * ни на что не влияют в противном случае
        % \nocite{vakbib2}%vak
        % \nocite{otherbib1}%other
        % \nocite{ptitt18}%conf
    \end{refsection}%
        %
        % Всё, что вне этих двух refsection, это refsection=0,
        %  * для диссертации - это нормальные ссылки, попадающие в обычную библиографию
        %  * для автореферата:
        %     * при usefootcite==0, ссылка корректно сработает только для источника из `external.bib`. Для своих работ --- напечатает "[0]" (и даже Warning не вылезет).
        %     * при usefootcite==1, ссылка сработает нормально. В авторской библиографии будут только процитированные в refsection=0 работы.
        %
        % Невидимый библиографический список для подсчёта количества внешних публикаций
        % Используется, чтобы убрать приставку "А" у работ автора, если в автореферате нет
        % цитирований внешних источников.
        % Замедляет компиляцию
    \ifsynopsis
    \ifnumequal{\value{draft}}{0}{
      \printbibliography[heading=nobibheading, section=0, env=countexternal,          keyword=biblioexternal]%
    }{}
    \fi
}

 % Характеристика работы по структуре во введении и в автореферате не отличается (ГОСТ Р 7.0.11, пункты 5.3.1 и 9.2.1), потому её загружаем из одного и того же внешнего файла, предварительно задав форму выделения некоторым параметрам

\textbf{Объем и структура работы.} Диссертация состоит из~введения, трёх глав,
заключения и~двух приложений.
%% на случай ошибок оставляю исходный кусок на месте, закомментированным
%Полный объём диссертации составляет  \ref*{TotPages}~страницу
%с~\totalfigures{}~рисунками и~\totaltables{}~таблицами. Список литературы
%содержит \total{citenum}~наименований.
%
Полный объём диссертации составляет
\formbytotal{TotPages}{страниц}{у}{ы}{}, включая
\formbytotal{totalcount@figure}{рисун}{ок}{ка}{ков} и
\formbytotal{totalcount@table}{таблиц}{у}{ы}{}.   Список литературы содержит
\formbytotal{citenum}{наименован}{ие}{ия}{ий}.
