\chapter{Оценивание параметров многомерных сигналов в инфокоммуникационных системах}\label{ch:ch1}

\section{Многомерные сигналы в инфокоммуникационных системах связи}\label{sec:ch1/sec1}

Большинство современных инфокоммуникационных систем имеют дискретную (цифровую) природу. Дискретизация во временной области, лежащая в основе цифровых систем связи, теперь стала столь же распространённой в частотной и пространственной областях.

Появление систем с множеством поднесущих - например, систем с Ортогональным Частотным разделением Каналов (ОЧРК, англ. Orthogonal Division Multiplexing - OFDM) и их различных модификаций - привнесло дискретизацию передаваемых сигналов в частотной области, а развитие систем MIMO (системы с множеством входных и выходных портов, англ. Multiple Input Multiple Output - MIMO) привнесло дискретизацию в пространстве.

Таким образом, принимаемые/передаваемые сигналы $s(\ta,\fra,\spa)$ в современных инфокоммуникационных системах представляются как сложные многомерные функции многих переменных - времени $\ta$, частоты $\fra$ и положения в пространстве $\spa$.

Современные приёмные и передающие устройства систем связи позволяют измерять/формировать волновые процессы в конечном числе точек этого многомерного измеряемого пространства:
\begin{equation}
\label{eq:ch1:1}
s(\ti,\fri,\spi)=s(\ta,\fra,\spa)\at[\Bigg]{\substack{\ta=T\ti \\ f=\scs\fri \\ \spa=\bm{p}_{\spi} }}\in\compl
\end{equation}
где $\ti$, $\fri$, $\spi$ - индексы отсчётов по времени, частоте и пространству, соответственно; $\ts$ - период дискретизации во времени; $\scs$ - период дискретизации по частоте (расстояние между поднесущими); $\bm{p}_{\spi}$ - точки пространства, в которых находятся приёмные/передающие устройства).

Уравнение \eqref{eq:ch1:1} подразумевает равномерную (с одинаковым шагом) дискретизацию сигналов во временной и частотной области, рассматриваемую в данной работе.

Такие возможности измерения волновых процессов (электро-магнитных, акустических, гидроакустических, сейсмических и т.д.) одновременно в пространстве, времени и частоте приводят к необходимости формулирования новых математических моделей представления используемых сигналов.

Как видно из уравнения~\eqref{eq:ch1:1}, измеренные значения имеют трёх-мерную структуру, после дискретизации проще всего представляемую 3-х мерным массивом данных, или трёх-мерным тензором $\ten{S}$:
\begin{equation}
\label{eq:ch1:2}
[\ten{S}]_{\ti,\fri,\spi}=s(\ti,\fri,\spi)\in\compl^{\Nt\times \Nf \times \Ns}
\end{equation}
где $\Nt$, $\Nf$, $\Ns$ - количество отсчётов сигнала по времени, частоте и пространству, соответственно.

Стоит заметить, что излагаемые представления в равной степени относятся и к системам, работающим с другими типами волновых процессов - например, с акустическими, гидроакустическими, сейсмическими и т.д. Это означает, что разрабатываемые алгоритмы также применимы и в этих других областях техники.

Тензорное представление исследуемых сигналов позволяет не только упростить запись математических операций, но и легче выявлять скрытую структуру данных, позволяющую применять тензорные разложения (например, такие как \cite{Bros99, Roemer12, Roemer13}) для оценки скрытых параметров моделей. Тензорные разложения, в отличие от их эквивалентов для матриц, обладают свойствами уникальности и идентифицируемости при выполнении более мягких требований.

\section{Параметрические каналы связи.}\label{sec:ch1/sec2}

Подробное теоретическое описание параметрических каналов можно найти в таких работах как \cite{Richter2005}. Приведём здесь краткое изложение модели параметрического канала связи, используемой в данной работе.

Любая инфокоммуникационная система обладает ещё большим количеством степеней свободы, так как рассматривается система с множеством выходных (приёмных) и входных (передающих) портов. Общепринятым методом описания процесса распространения сигнала от передатчика к приёмнику является описание через передаточную характеристику канала связи $\ten{H}$. В системах MIMO, в общем виде, помимо времени и частоты, передаточная функция указывается для всех комбинаций входных и выходных портом (антенн), иными словами, может быть представлена как минимум 4-х мерным тензором.

Например, в общем виде отсчёты принятых сигналов систем MIMO можно записать как:
\begin{equation}
\label{eq:ch1:3}
y(\ti,\fri,\rxi) = \sum_{\txi=1}^{\Ntx} h(\ti,\fri,\rxi,\txi)\cdot s(\ti,\fri,\txi) + n(\ti,\fri,\rxi)
\end{equation}
где $0\le\rxi<\Nrx$ и $0\le\txi<\Ntx$ - индексы передающих и приёмных портов (антенн), соответственно; $\Ntx$, $\Nrx$ - количество передающих и приёмных портов (антенн); $s(\ti,\fri,\txi)$ - переданные сигналы с каждой антенны; $n(\ti,\fri,\rxi)$ - аддитивная помеха/шум.

Таким образом, передаточная характеристика канала может быть также представлена в виде тензора:
\begin{equation}
\label{eq:ch1:4}
[\ten{H}]_{\ti,\fri,\rxi,\txi}=h(\ti,\fri,\rxi,\txi)\in\compl^{\Nt\times \Nf \times \Nrx \times \Ntx}
\end{equation}

В частных случаях расположений передающих/приёмных портов, передаточная характеристика может иметь больше измерений. Например, при использовании прямоугольных антенных массивов (англ. Uniform Rectangular Array - URA), пространственные измерения могут быть дополнительно разбиты на измерения соответствующие осям плоскости массивов антенн.

Передаточная характеристика канала играет ключевую роль при восстановлении (оценивании) переданного сигнала $s()$ на стороне приёмника по искажённому принятому сигналу $y()$, поэтому одной из основных задач в любой инфокоммуникационной системе является \textit{эквализация} канала связи - т.е. удаление его влияния на переданный сигнал. 

В простейшем случае канал связи может быть оценён в процессе \textit{измерения канала связи} - т.е. когда переданный сигнал известен на приёмной стороне и используется для оценки канала. Однако, как видно из уравнения \eqref{eq:ch1:4} количество неизвестных $N=\Nt\Nf\Nrx\Ntx$ геометрически возрастает с увеличением количества используемых системой связи ресурсов (полосы частот, приёмных или передающих антенн), поэтому представление канала как полностью неизвестной величины является неэффективным решением.

Поэтому широкое распространение получили модели канала связи, использующие внутреннюю структуру канала для экономного его описания с помощью гораздо меньшего количества скрытых параметров. Такие каналы можно назвать параметрическими, а вектор неизвестных скрытых параметров в данном случае обозначается как $\pars\in\real^\Npars$ и содержит конечное количество вещественных параметров $\Npars$.

\subsection{Многолучевая модель канала связи}\label{subsec:ch1/sec1/sub1}

Самым распространённым подходом к моделированию канала связи является использование многолучевой модели распространения сигналов. В этом случае канал связи на каждой поднесущей $\fri$ и в каждый момент времени $\ti$ представляется как сумма конечного количества лучей, идущих от каждой передающей антенны к каждой приёмной.

Для использования такого описания канала связи примем следующие предположения:
\begin{assumption}
\label{as:ch1:1}
На каждой поднесущей $\fri$ и в каждый момент времени $\ti$, канал связи представляется как суперпозиция узкополосных каналов передачи (лучей) с постоянной частотной передаточной характеристикой.
\end{assumption}

Канал с постоянной частотной передаточной характеристикой во временной области имеет импульсную характеристику в виде смещённой дельта-функции $g(t)=A\sigma (t-\tau)$, что, в свою очередь эквивалентно описанию передаточной характеристики с помощью одного комплексного коэффициента $h=Ae^{-j\phi}$. 

Определим также множество $\set{L}$ индексов всех лучей канала связи от каждой передающей антенны $n$ к массиву приёмных антенн, а множество индексов всех лучей, идущих от $\txi$-й антенны обозначим как $\set{L}_n$. Следовательно, можно записать:
\begin{equation}
\label{eq:ch1:5}
\set{L} = \set{L}_0 \cup ... \cup \set{L}_{\Ntx-1}
\end{equation}

Таким образом, предположение~\ref{as:ch1:1} позволяет описывать канал связи всей системы как:
\begin{equation}
\label{eq:ch1:6}
h(\ti,\fri,\rxi,\txi)=\sum_{\pai\in\Spai} \hpa(\ti,\fri,\rxi,\txi)
\end{equation}
где $\pai\in\Spai$ - индекс пути распространения (луча) от передающей антенны $\txi$ к массиву приёмных антенн. Обратите внимание,  что в общем виде каждая антенна имеет $\Npa=|\Spai|$ независимых дискретных путей к массиву приёмных антенн. В частных случаях эти пути могут практически совпадать друг с другом, например, если передающие антенны находятся близко друг к другу (массив передающих антенн).
	
Коэффициент передачи одного пути $\hpa$ можно записать следующим образом:
\begin{equation}
\label{eq:ch1:7}
\hpa(\ti,\fri,\rxi,\txi) = \tgaini^{\ti,\fri,\rxi} \cdot 
                           e^{-j2\pi \sfr \fulldelay} \cdot 
                           e^{j2\pi \frac{\sfr \ts\ti}{c} \vrel^\rxi}
\end{equation}
где:
\begin{description}
	\item[$c$] - скорость распространения волны в среде (например, скорость света);
	\item[$\lfr$] - нижняя граничная частота системы (наименьшая частота поднесущей);
	\item[$\sfr=\lfr + \scs \fri$] - частота поднесущей;
	\item[$\tgaini^{\ti,\fri,\rxi}$] - комплексный коэффициент затухания сигнала по пути $\pai$ на $\fri$-й частоте в $\ti$-й момент времени к $\rxi$-й антенне, включающий эффекты затухания и сдвига фазы в среде распространения и при отражениях/рассеиваниях;
	\item[$\vrel^\rxi$] - относительная скорость движения по пути $\pai$ к антенне $\rxi$;
	\item[$\fulldelay$] - полная задержка распространения от антенны $\txi$ по пути $\pai$ к антенне $\rxi$.
\end{description}

Как видно из уравнения \eqref{eq:ch1:7}, передаточный коэффициент $h_\pai$ представляет несколько физических эффектов, влияющих на распространение сигналов:
\begin{enumerate}
	\item Затухания в среде и отражения/рассеивания от границ разделов разных сред
	\item Сдвиг фазы волны в результате распространения по пути луча
	\item Сдвиг фазы волны в результате эффекта Допплера (смещение Допплера), появляющийся в результате движения передающих и приёмных антенн друг относительно друга
\end{enumerate}

Для упрощения модели данных в системах MIMO справедливы следующие предположения:
\begin{assumption}
	\label{as:ch1:2}
	Относительная скорость вдоль любого луча любой передающей антенны одинакова для всех приёмных антенн, т.е. $\vrel^\rxi=\vrel$ $\forall$ $0\leq\rxi<\Nrx$.
\end{assumption}
\begin{assumption}
\label{as:ch1:3}
Коэффициент затухания сигнала одинаков для всех приёмных антенн, т.е. $\tgaini^{\ti,\fri,\rxi}=\tgaini^{\ti,\fri}$ $\forall$ $0\leq\rxi<\Nrx$.
\end{assumption}
\begin{assumption}
	\label{as:ch1:4}
	Коэффициент затухания сигнала не зависит от частоты, т.е. $\tgaini^{\ti,\fri}=\tgaini^{\ti}$ $\forall$ $0\leq\fri<\Nf$.
\end{assumption}

Предположение~\ref{as:ch1:2} говорит о том, что относительная скорость вдоль луча $\vrel$ не зависит от индекса приёмной антенны $\rxi$.

Предположения~\ref{as:ch1:2} и \ref{as:ch1:3} справедливы в случае когда приёмные антенны расположены рядом друг с другом, и расстояние между ними много меньше длин лучей $\di$, что чаще всего соблюдается в MIMO системах.

Обратите внимание, что верхний индекс $\txi$ величины $\fulldelay$ необязательно должен равняться индексу $\txi$ в $\pai$, т.е. величина $\fulldelay$ может определять задержку антенны относительно пути для другой антенны. Например, величина $\tilde{\tau}_{l_0=0}^{2,2}$ обозначает задержку от передающей антенны $\txi=2$ по пути $l_0=0$ к приёмной антенне $\rxi=2$.

Тогда, задержку распространения $\fulldelay$ можно представить в виде:
\begin{equation}
\label{eq:ch1:8}
\fulldelay = \tilde{\tau}^{\txi,0}_{\pai} + \frac{\delrx}{c}
\end{equation}
или
\begin{equation}
\label{eq:ch1:9}
\fulldelay = \tilde{\tau}^{0,0}_{\pai} + \frac{\deltx + \delrx}{c}
\end{equation}
где:
\begin{description}
	\item[$d_{\pai}^\txi$] - расстояние между $\txi$-й передающей и опорной приёмной антеннами по пути луча $\pai$;	
	\item[$\tilde{\tau}^{\txi,0}_{\pai}=\delay^\txi=d_{\pai}^\txi/c$] - задержка распространения между $\txi$-й передающей и опорной приёмной антеннами по пути луча $\pai$;
	\item[$\di$] - расстояние между опорной передающей и опорной приёмной антеннами по пути луча $\pai$;
	\item[$\tilde{\tau}^{0,0}_{\pai}=\delay=\di/c$] - задержка распространения между опорной передающей и опорной приёмной антеннами по пути луча $\pai$;
	\item[$\deltx$] - геометрическая разность хода фронта волны от $\txi$-й передающей антенны до <<опорной>> передающей антенны (с индексом 0) по пути луча $\pai$;
	\item[$\delrx$] - геометрическая разность хода фронта волны от $\rxi$-й приёмной антенны до <<опорной>> приёмной антенны (с индексом 0) по пути луча $\pai$;
\end{description}

Таким образом, учитывая предположение~\ref{as:ch1:2} и используя уравнение \eqref{eq:ch1:8}, уравнение~\eqref{eq:ch1:7} можно переписать в виде:
\begin{equation}
\begin{aligned}
\label{eq:ch1:10}
\hpa(\ti,\fri,\rxi,\txi) =&\underbrace{\tgaini^{\ti} \cdot e^{-j2\pi \lfr \delay^\txi}}_{\gain_{\pai,\txi}^\ti} \cdot 
						   \underbrace{e^{-j2\pi \scs \fri \delay^\txi}}_{b_{\pai,\fri,\txi}^f} \cdot
						   \underbrace{e^{j2\pi \frac{\sfr \ts\ti}{c} \vrel}}_{b_{\pai,\ti,\fri}^t} \cdot
						   \underbrace{e^{-j2\pi \frac{\sfr}{c} \delrx}}_{b_{\pai,\rxi,\fri}^\te{rx}} \\
						 =&\gain_{\pai,\txi}^\ti \cdot b_{\pai,\fri,\txi}^f \cdot b_{\pai,\ti,\fri}^t  \cdot b_{\pai,\rxi,\fri}^\te{rx}
\end{aligned}
\end{equation}
или используя уравнение \eqref{eq:ch1:9}:
\begin{equation}
\begin{aligned}
\label{eq:ch1:11}
\hpa(\ti,\fri,\rxi,\txi) =&\underbrace{\tgaini^{\ti} \cdot e^{-j2\pi \lfr \delay}}_\gaini \cdot 
						   \underbrace{e^{-j2\pi \scs \fri \delay}}_{b_{\pai,\fri}^f} \cdot
					       \underbrace{e^{j2\pi \frac{\sfr \ts\ti}{c} \vrel}}_{b_{\pai,\ti,\fri}^t} \cdot
						   \underbrace{e^{-j2\pi \frac{\sfr}{c} \delrx}}_{b_{\pai,\rxi,\fri}^\te{rx}} \cdot
						   \underbrace{e^{-j2\pi \frac{\sfr}{c} \deltx}}_{b_{\pai,\txi,\fri}^\te{tx}} \\
						 =&\gaini \cdot b_{\pai,\fri}^f \cdot b_{\pai,\ti,\fri}^t \cdot b_{\pai,\rxi,\fri}^\te{rx} \cdot b_{\pai,\txi,\fri}^\te{tx}
\end{aligned}
\end{equation}

Представление~\eqref{eq:ch1:10} подходит для систем с разнесёнными (не объединёнными в один массив) передатчиками. Например, если рассматривается восходящая линия мобильной сети, в которой каждая передающая антенна это один абонент сети.

Представление~\eqref{eq:ch1:11}, напротив, подходит для случая системы MIMO с массивом близкорасположенных передающих антенн. В этом случае удобнее выразить задержки относительно одной <<опорной>> передающей антенны.

\begin{comment}
% \cite[с.~54]{Sokolov}\cite[с.~36]{Gaidaenko}.
% Две ссылки: \cite{Sokolov,Gaidaenko}.
% Ссылка на собственные работы: \cite{vakbib1, confbib2}.
% Много ссылок: %\cite[с.~54]{Lermontov,Management,Borozda} % такой «фокус»
%вызывает biblatex warning относительно опции sortcites, потому что неясно, к
%какому источнику относится уточнение о страницах, а bibtex об этой проблеме
%даже не предупреждает
% \cite{Lermontov, Management, Borozda, Marketing, Constitution, FamilyCode,
% Gost.7.0.53, Razumovski, Lagkueva, Pokrovski, Methodology, Nasirova, Berestova,
% Kriger}%
% \ifnumequal{\value{bibliosel}}{0}{% Примеры для bibtex8
%     \cite{Sirotko, Lukina, Encyclopedia}%
% }{% Примеры для biblatex через движок biber
%     \cite{Sirotko2, Lukina2, Encyclopedia2}%
% }%


\section{Формулы}\label{sec:ch1/sec4}

Благодаря пакету \textit{icomma}, \LaTeX~одинаково хорошо воспринимает
в~качестве десятичного разделителя и запятую (\(3,1415\)), и точку (\(3.1415\)).

\subsection{Ненумерованные одиночные формулы}\label{subsec:ch1/sec4/sub1}

Вот так может выглядеть формула, которую необходимо вставить в~строку
по~тексту: \(x \approx \sin x\) при \(x \to 0\).

А вот так выглядит ненумерованная отдельностоящая формула c подстрочными
и надстрочными индексами:
\[
(x_1+x_2)^2 = x_1^2 + 2 x_1 x_2 + x_2^2
\]

Формула с неопределенным интегралом:
\[
\int f(\alpha+x)=\sum\beta
\]

При использовании дробей формулы могут получаться очень высокие:
\[
  \frac{1}{\sqrt{2}+
  \displaystyle\frac{1}{\sqrt{2}+
  \displaystyle\frac{1}{\sqrt{2}+\cdots}}}
\]

В формулах можно использовать греческие буквы:
%Все \original... команды заранее, ради этого примера, определены в Dissertation\userstyles.tex
\[
\alpha\beta\gamma\delta\originalepsilon\epsilon\zeta\eta\theta%
\vartheta\iota\kappa\varkappa\lambda\mu\nu\xi\pi\varpi\rho\varrho%
\sigma\varsigma\tau\upsilon\originalphi\phi\chi\psi\omega\Gamma\Delta%
\Theta\Lambda\Xi\Pi\Sigma\Upsilon\Phi\Psi\Omega
\]
\[%https://texfaq.org/FAQ-boldgreek
\boldsymbol{\alpha\beta\gamma\delta\originalepsilon\epsilon\zeta\eta%
\theta\vartheta\iota\kappa\varkappa\lambda\mu\nu\xi\pi\varpi\rho%
\varrho\sigma\varsigma\tau\upsilon\originalphi\phi\chi\psi\omega\Gamma%
\Delta\Theta\Lambda\Xi\Pi\Sigma\Upsilon\Phi\Psi\Omega}
\]

Для добавления формул можно использовать пары \verb+$+\dots\verb+$+ и \verb+$$+\dots\verb+$$+,
но~они считаются устаревшими.
Лучше использовать их функциональные аналоги \verb+\(+\dots\verb+\)+ и \verb+\[+\dots\verb+\]+.

\subsection{Ненумерованные многострочные формулы}\label{subsec:ch1/sec4/sub2}

Вот так можно написать две формулы, не нумеруя их, чтобы знаки <<равно>> были
строго друг под другом:
\begin{align}
  f_W & =  \min \left( 1, \max \left( 0, \frac{W_{soil} / W_{max}}{W_{crit}} \right)  \right), \nonumber \\
  f_T & =  \min \left( 1, \max \left( 0, \frac{T_s / T_{melt}}{T_{crit}} \right)  \right), \nonumber
\end{align}

Выровнять систему ещё и по переменной \( x \) можно, используя окружение
\verb|alignedat| из пакета \verb|amsmath|. Вот так:
\[
    |x| = \left\{
    \begin{alignedat}{2}
        &&x, \quad &\text{eсли } x\geqslant 0 \\
        &-&x, \quad & \text{eсли } x<0
    \end{alignedat}
    \right.
\]
Здесь первый амперсанд (в исходном \LaTeX\ описании формулы) означает
выравнивание по~левому краю, второй "--- по~\( x \), а~третий "--- по~слову
<<если>>. Команда \verb|\quad| делает большой горизонтальный пробел.

Ещё вариант:
\[
    |x|=
    \begin{cases}
    \phantom{-}x, \text{если } x \geqslant 0 \\
    -x, \text{если } x<0
    \end{cases}
\]

Кроме того, для  нумерованных формул \verb|alignedat| делает вертикальное
выравнивание номера формулы по центру формулы. Например, выравнивание
компонент вектора:
\begin{equation}
\label{eq:2p3}
\begin{alignedat}{2}
{\mathbf{N}}_{o1n}^{(j)} = \,{\sin} \phi\,n\!\left(n+1\right)
         {\sin}\theta\,
         \pi_n\!\left({\cos} \theta\right)
         \frac{
               z_n^{(j)}\!\left( \rho \right)
              }{\rho}\,
           &{\boldsymbol{\hat{\mathrm e}}}_{r}\,+   \\
+\,
{\sin} \phi\,
         \tau_n\!\left({\cos} \theta\right)
         \frac{
            \left[\rho z_n^{(j)}\!\left( \rho \right)\right]^{\prime}
              }{\rho}\,
            &{\boldsymbol{\hat{\mathrm e}}}_{\theta}\,+   \\
+\,
{\cos} \phi\,
         \pi_n\!\left({\cos} \theta\right)
         \frac{
            \left[\rho z_n^{(j)}\!\left( \rho \right)\right]^{\prime}
              }{\rho}\,
            &{\boldsymbol{\hat{\mathrm e}}}_{\phi}\:.
\end{alignedat}
\end{equation}

Ещё об отступах. Иногда для лучшей <<читаемости>> формул полезно
немного исправить стандартные интервалы \LaTeX\ с учётом логической
структуры самой формулы. Например в формуле~\ref{eq:2p3} добавлен
небольшой отступ \verb+\,+ между основными сомножителями, ниже
результат применения всех вариантов отступа:
\begin{align*}
\backslash! &\quad f(x) = x^2\! +3x\! +2 \\
  \mbox{по-умолчанию} &\quad f(x) = x^2+3x+2 \\
\backslash, &\quad f(x) = x^2\, +3x\, +2 \\
\backslash{:} &\quad f(x) = x^2\: +3x\: +2 \\
\backslash; &\quad f(x) = x^2\; +3x\; +2 \\
\backslash \mbox{space} &\quad f(x) = x^2\ +3x\ +2 \\
\backslash \mbox{quad} &\quad f(x) = x^2\quad +3x\quad +2 \\
\backslash \mbox{qquad} &\quad f(x) = x^2\qquad +3x\qquad +2
\end{align*}

Можно использовать разные математические алфавиты:
\begin{align}
\mathcal{ABCDEFGHIJKLMNOPQRSTUVWXYZ} \nonumber \\
\mathfrak{ABCDEFGHIJKLMNOPQRSTUVWXYZ} \nonumber \\
\mathbb{ABCDEFGHIJKLMNOPQRSTUVWXYZ} \nonumber
\end{align}

Посмотрим на систему уравнений на примере аттрактора Лоренца:

\[
\left\{
  \begin{array}{rl}
    \dot x = & \sigma (y-x) \\
    \dot y = & x (r - z) - y \\
    \dot z = & xy - bz
  \end{array}
\right.
\]

А для вёрстки матриц удобно использовать многоточия:
\[
\left(
  \begin{array}{ccc}
    a_{11} & \ldots & a_{1n} \\
    \vdots & \ddots & \vdots \\
    a_{n1} & \ldots & a_{nn} \\
  \end{array}
\right)
\]

\subsection{Нумерованные формулы}\label{subsec:ch1/sec4/sub3}

А вот так пишется нумерованная формула:
\begin{equation}
  \label{eq:equation1}
  e = \lim_{n \to \infty} \left( 1+\frac{1}{n} \right) ^n
\end{equation}

Нумерованных формул может быть несколько:
\begin{equation}
  \label{eq:equation2}
  \lim_{n \to \infty} \sum_{k=1}^n \frac{1}{k^2} = \frac{\pi^2}{6}
\end{equation}

Впоследствии на формулы~\eqref{eq:equation1} и~\eqref{eq:equation2} можно ссылаться.

Сделать так, чтобы номер формулы стоял напротив средней строки, можно,
используя окружение \verb|multlined| (пакет \verb|mathtools|) вместо
\verb|multline| внутри окружения \verb|equation|. Вот так:
\begin{equation} % \tag{S} % tag - вписывает свой текст
  \label{eq:equation3}
    \begin{multlined}
        1+ 2+3+4+5+6+7+\dots + \\
        + 50+51+52+53+54+55+56+57 + \dots + \\
        + 96+97+98+99+100=5050
    \end{multlined}
\end{equation}

Используя команду \verb|\eqrefs|, можно
красиво ссылаться сразу на несколько формул
\eqrefs{eq:equation1, eq:equation3, eq:equation2}, даже перепутав
порядок ссылок \verb|\eqrefs{eq1, eq3, eq2}|.
Аналогично, для ссылок на несколько рисунков, таблиц и~т.\:д.
\refs{sec:ch1/sec1, sec:ch1/sec2, sec:ch1/sec3} можно использовать
команду \verb|\refs|.
Обе эти команды определены в файле \verb|common/packages.tex|.

Уравнения~\eqrefs{eq:subeq_1,eq:subeq_2} демонстрируют возможности
окружения \verb|\subequations|.
\begin{subequations}
    \label{eq:subeq_1}
    \begin{gather}
        y = x^2 + 1 \label{eq:subeq_1-1} \\
        y = 2 x^2 - x + 1 \label{eq:subeq_1-2}
    \end{gather}
\end{subequations}
Ссылки на отдельные уравнения~\eqrefs{eq:subeq_1-1,eq:subeq_1-2,eq:subeq_2-1}.
\begin{subequations}
    \label{eq:subeq_2}
    \begin{align}
        y &= x^3 + x^2 + x + 1 \label{eq:subeq_2-1} \\
        y &= x^2
    \end{align}
\end{subequations}

\subsection{Форматирование чисел и размерностей величин}\label{sec:units}

Числа форматируются при помощи команды \verb|\num|:
\num{5,3};
\num{2,3e8};
\num{12345,67890};
\num{2,6 d4};
\num{1+-2i};
\num{.3e45};
\num[exponent-base=2]{5 e64};
\num[exponent-base=2,exponent-to-prefix]{5 e64};
\num{1.654 x 2.34 x 3.430}
\num{1 2 x 3 / 4}.
Для написания последовательности чисел можно использовать команды \verb|\numlist| и \verb|\numrange|:
\numlist{10;30;50;70}; \numrange{10}{30}.
Значения углов можно форматировать при помощи команды \verb|\ang|:
\ang{2.67};
\ang{30,3};
\ang{-1;;};
\ang{;-2;};
\ang{;;-3};
\ang{300;10;1}.

Обратите внимание, что ГОСТ запрещает использование знака <<->> для обозначения отрицательных чисел
за исключением формул, таблиц и~рисунков.
Вместо него следует использовать слово <<минус>>.

Размерности можно записывать при помощи команд \verb|\si| и \verb|\SI|:
\si{\farad\squared\lumen\candela};
\si{\joule\per\mole\per\kelvin};
\si[per-mode = symbol-or-fraction]{\joule\per\mole\per\kelvin};
\si{\metre\per\second\squared};
\SI{0.10(5)}{\neper};
\SI{1.2-3i e5}{\joule\per\mole\per\kelvin};
\SIlist{1;2;3;4}{\tesla};
\SIrange{50}{100}{\volt}.
Список единиц измерений приведён в таблицах~\refs{tab:unit:base,
tab:unit:derived,tab:unit:accepted,tab:unit:physical,tab:unit:other}.
Приставки единиц приведены в~таблице~\ref{tab:unit:prefix}.

С дополнительными опциями форматирования можно ознакомиться в~описании пакета \texttt{siunitx};
изменить или добавить единицы измерений можно в~файле \texttt{siunitx.cfg}.

\begin{table}
    \centering
    \captionsetup{justification=centering} % выравнивание подписи по-центру
    \caption{Основные величины СИ}\label{tab:unit:base}
    \begin{tabular}{llc}
        \toprule
        Название  & Команда                & Символ         \\
        \midrule
        Ампер     & \verb|\ampere| & \si{\ampere}   \\
        Кандела   & \verb|\candela| & \si{\candela}  \\
        Кельвин   & \verb|\kelvin| & \si{\kelvin}   \\
        Килограмм & \verb|\kilogram| & \si{\kilogram} \\
        Метр      & \verb|\metre| & \si{\metre}    \\
        Моль      & \verb|\mole| & \si{\mole}     \\
        Секунда   & \verb|\second| & \si{\second}   \\
        \bottomrule
    \end{tabular}
\end{table}

\begin{table}
  \small
  \centering
  \begin{threeparttable}% выравнивание подписи по границам таблицы
    \caption{Производные единицы СИ}\label{tab:unit:derived}
    \begin{tabular}{llc|llc}
        \toprule
        Название       & Команда                 & Символ              & Название & Команда & Символ \\
        \midrule
        Беккерель      & \verb|\becquerel|  & \si{\becquerel}     &
        Ньютон         & \verb|\newton|  & \si{\newton}                                      \\
        Градус Цельсия & \verb|\degreeCelsius| & \si{\degreeCelsius} &
        Ом             & \verb|\ohm| & \si{\ohm}                                         \\
        Кулон          & \verb|\coulomb| & \si{\coulomb}       &
        Паскаль        & \verb|\pascal| & \si{\pascal}                                      \\
        Фарад          & \verb|\farad| & \si{\farad}         &
        Радиан         & \verb|\radian| & \si{\radian}                                      \\
        Грей           & \verb|\gray| & \si{\gray}          &
        Сименс         & \verb|\siemens| & \si{\siemens}                                     \\
        Герц           & \verb|\hertz| & \si{\hertz}         &
        Зиверт         & \verb|\sievert| & \si{\sievert}                                     \\
        Генри          & \verb|\henry| & \si{\henry}         &
        Стерадиан      & \verb|\steradian| & \si{\steradian}                                   \\
        Джоуль         & \verb|\joule| & \si{\joule}         &
        Тесла          & \verb|\tesla| & \si{\tesla}                                       \\
        Катал          & \verb|\katal| & \si{\katal}         &
        Вольт          & \verb|\volt| & \si{\volt}                                        \\
        Люмен          & \verb|\lumen| & \si{\lumen}         &
        Ватт           & \verb|\watt| & \si{\watt}                                        \\
        Люкс           & \verb|\lux| & \si{\lux}           &
        Вебер          & \verb|\weber| & \si{\weber}                                       \\
        \bottomrule
    \end{tabular}
  \end{threeparttable}
\end{table}

\begin{table}
  \centering
  \begin{threeparttable}% выравнивание подписи по границам таблицы
    \caption{Внесистемные единицы}\label{tab:unit:accepted}

    \begin{tabular}{llc}
        \toprule
        Название        & Команда                 & Символ          \\
        \midrule
        День            & \verb|\day| & \si{\day}       \\
        Градус          & \verb|\degree| & \si{\degree}    \\
        Гектар          & \verb|\hectare| & \si{\hectare}   \\
        Час             & \verb|\hour| & \si{\hour}      \\
        Литр            & \verb|\litre| & \si{\litre}     \\
        Угловая минута  & \verb|\arcminute| & \si{\arcminute} \\
        Угловая секунда & \verb|\arcsecond| & \si{\arcsecond} \\ %
        Минута          & \verb|\minute| & \si{\minute}    \\
        Тонна           & \verb|\tonne| & \si{\tonne}     \\
        \bottomrule
    \end{tabular}
  \end{threeparttable}
\end{table}

\begin{table}
    \centering
    \captionsetup{justification=centering}
    \caption{Внесистемные единицы, получаемые из эксперимента}\label{tab:unit:physical}
    \begin{tabular}{llc}
        \toprule
        Название                & Команда                 & Символ                 \\
        \midrule
        Астрономическая единица & \verb|\astronomicalunit| & \si{\astronomicalunit} \\
        Атомная единица массы   & \verb|\atomicmassunit| & \si{\atomicmassunit}   \\
        Боровский радиус        & \verb|\bohr| & \si{\bohr}             \\
        Скорость света          & \verb|\clight| & \si{\clight}           \\
        Дальтон                 & \verb|\dalton| & \si{\dalton}           \\
        Масса электрона         & \verb|\electronmass| & \si{\electronmass}     \\
        Электрон Вольт          & \verb|\electronvolt| & \si{\electronvolt}     \\
        Элементарный заряд      & \verb|\elementarycharge| & \si{\elementarycharge} \\
        Энергия Хартри          & \verb|\hartree| & \si{\hartree}          \\
        Постоянная Планка       & \verb|\planckbar| & \si{\planckbar}        \\
        \bottomrule
    \end{tabular}
\end{table}

\begin{table}
  \centering
  \begin{threeparttable}% выравнивание подписи по границам таблицы
    \caption{Другие внесистемные единицы}\label{tab:unit:other}
    \begin{tabular}{llc}
        \toprule
        Название                  & Команда                 & Символ             \\
        \midrule
        Ангстрем                  & \verb|\angstrom| & \si{\angstrom}     \\
        Бар                       & \verb|\bar| & \si{\bar}          \\
        Барн                      & \verb|\barn| & \si{\barn}         \\
        Бел                       & \verb|\bel| & \si{\bel}          \\
        Децибел                   & \verb|\decibel| & \si{\decibel}      \\
        Узел                      & \verb|\knot| & \si{\knot}         \\
        Миллиметр ртутного столба & \verb|\mmHg| & \si{\mmHg}         \\
        Морская миля              & \verb|\nauticalmile| & \si{\nauticalmile} \\
        Непер                     & \verb|\neper| & \si{\neper}        \\
        \bottomrule
    \end{tabular}
  \end{threeparttable}
\end{table}

\begin{table}
  \small
  \centering
  \begin{threeparttable}% выравнивание подписи по границам таблицы
    \caption{Приставки СИ}\label{tab:unit:prefix}
    \begin{tabular}{llcc|llcc}
        \toprule
        Приставка & Команда                 & Символ      & Степень &
        Приставка & Команда                 & Символ      & Степень   \\
        \midrule
        Иокто     & \verb|\yocto| & \si{\yocto} & -24     &
        Дека      & \verb|\deca| & \si{\deca}  & 1         \\
        Зепто     & \verb|\zepto| & \si{\zepto} & -21     &
        Гекто     & \verb|\hecto| & \si{\hecto} & 2         \\
        Атто      & \verb|\atto| & \si{\atto}  & -18     &
        Кило      & \verb|\kilo| & \si{\kilo}  & 3         \\
        Фемто     & \verb|\femto| & \si{\femto} & -15     &
        Мега      & \verb|\mega| & \si{\mega}  & 6         \\
        Пико      & \verb|\pico| & \si{\pico}  & -12     &
        Гига      & \verb|\giga| & \si{\giga}  & 9         \\
        Нано      & \verb|\nano| & \si{\nano}  & -9      &
        Терра     & \verb|\tera| & \si{\tera}  & 12        \\
        Микро     & \verb|\micro| & \si{\micro} & -6      &
        Пета      & \verb|\peta| & \si{\peta}  & 15        \\
        Милли     & \verb|\milli| & \si{\milli} & -3      &
        Екса      & \verb|\exa| & \si{\exa}   & 18        \\
        Санти     & \verb|\centi| & \si{\centi} & -2      &
        Зетта     & \verb|\zetta| & \si{\zetta} & 21        \\
        Деци      & \verb|\deci| & \si{\deci}  & -1      &
        Иотта     & \verb|\yotta| & \si{\yotta} & 24        \\
        \bottomrule
    \end{tabular}
  \end{threeparttable}
\end{table}

\subsection{Заголовки с формулами: \texorpdfstring{\(a^2 + b^2 = c^2\)}{%
a\texttwosuperior\ + b\texttwosuperior\ = c\texttwosuperior},
\texorpdfstring{\(\left\vert\textrm{{Im}}\Sigma\left(
\protect\varepsilon\right)\right\vert\approx const\)}{|ImΣ (ε)| ≈ const},
\texorpdfstring{\(\sigma_{xx}^{(1)}\)}{σ\_\{xx\}\textasciicircum\{(1)\}}
}\label{subsec:with_math}

Пакет \texttt{hyperref} берёт текст для закладок в pdf-файле из~аргументов
команд типа \verb|\section|, которые могут содержать математические формулы,
а~также изменения цвета текста или шрифта, которые не отображаются в~закладках.
Чтобы использование формул в заголовках не вызывало в~логе компиляции появление
предупреждений типа <<\texttt{Token not allowed in~a~PDF string
(Unicode):(hyperref) removing...}>>, следует использовать конструкцию
\verb|\texorpdfstring{}{}|, где в~первых фигурных скобках указывается
формула, а~во~вторых "--- запись формулы для закладок.

\section{Рецензирование текста}\label{sec:markup}

В шаблоне для диссертации и автореферата заданы команды рецензирования.
Они видны при компиляции шаблона в режиме черновика или при установке
соответствующей настройки (\verb+showmarkup+) в~файле \verb+common/setup.tex+.

Команда \verb+\todo+ отмечает текст красным цветом.
\todo{Например, так.}

Команда \verb+\note+ позволяет выбрать цвет текста.
\note{Чёрный, } \note[red]{красный, } \note[green]{зелёный, }
\note[blue]{синий.} \note[orange]{Обратите внимание на ширину и расстановку
формирующихся пробелов, в~результате приведённой записи (зависит также
от~применяемого компилятора).}

Окружение \verb+commentbox+ также позволяет выбрать цвет.

\begin{commentbox}[red]
        Красный текст.

        Несколько параграфов красного текста.
\end{commentbox}

\begin{commentbox}[blue]
        Синяя формула.

        \begin{equation}
                \alpha + \beta = \gamma
        \end{equation}
\end{commentbox}

\verb+commentbox+ позволяет закомментировать участок кода в~режиме чистовика.
Чтобы убрать кусок кода для всех режимов, можно использовать окружение
\verb+comment+.

        Этот текст всегда скрыт.
\end{comment}
