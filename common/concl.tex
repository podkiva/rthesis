%% Согласно ГОСТ Р 7.0.11-2011:
%% 5.3.3 В заключении диссертации излагают итоги выполненного исследования, рекомендации, перспективы дальнейшей разработки темы.
%% 9.2.3 В заключении автореферата диссертации излагают итоги данного исследования, рекомендации и перспективы дальнейшей разработки темы.

\begin{itemize}
\item предложен метод предварительной обработки многомерных сигналов в широкополосных системах, позволяющий применять алгоритмы оценивания параметров каналов связи, разработанные для узкополосных систем. Предложенный метод предварительной обработки многомерных сигналов в широкополосных системах существенно повышает точность алгоритмов оценки параметров каналов связи, разработанных для узкополосных систем. Улучшение точности оценивания увеличивается при увеличении относительной полосы частот рассматриваемой широкополосной системы. Например, показана эффективность данного метода в широкополосных системах с относительной полосой частот, превышающей 10\%; 
\item разработан новый алгоритм оценивания параметров канала связи с отражателями в ближнем геометрическом поле с использованием сферической модели фронта волны. С помощью компьютерного моделирования показано, что разработанный алгоритм оценивания параметров канала связи с отражателями в ближнем геометрическом поле более эффективен чем существующие алгоритмы;
\item впервые исследована граница Крамера-Рао для задач оценки параметров каналов связи с негауссовым распределением аддитивной помехи, заданным смесью нормальных распределений с ненулевыми средними значениями компонент. Показано, что использование более сложных моделей аддитивных помех, в частности с негауссовским распределением аддитивной помехи, выраженным смесью нормальных распределений, потенциально позволяет увеличить точность работы алгоритмов оценивания параметров каналов связи;
\item разработана программная реализация предложенных алгоритмов, а также компьютерные модели, имитирующие работу инфокоммуникационных систем с предложенными алгоритмами и оценивающие эффективность их работы.
\end{itemize}

\begin{comment}
Результаты математического моделирования показали практическую применимость метода вычисления нижней границы Крамера-Рао. Используя данный метод граница может быть вычислена при произвольном распределении аддитивной помехи. Результаты моделирования также показали, что граница Крамера-Рао ниже для сложных распределений аддитивных помех. Следовательно, алгоритмы на основе сложных моделей аддитивных помех содержат потенциал к улучшению качества оценивания искомых параметров, и, как следствие, к улучшению качественных показателей функционирования систем радиолокации и телекоммуникаций.
\end{comment}