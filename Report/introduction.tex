\chapter*{Введение}                         % Заголовок
\addcontentsline{toc}{chapter}{Введение}    % Добавляем его в оглавление

Во \underline{\textbf{введении}} обосновывается актуальность
исследований, проводимых в~рамках данной диссертационной работы,
приводится обзор научной литературы по~изучаемой проблеме,
формулируется цель, ставятся задачи работы, излагается научная новизна
и практическая значимость представляемой работы. 

\underline{\textbf{Первая глава}} описывает обобщённую теорию представления многомерных сигналов и каналов связи, их параметрическое представление. Затем, из общей параметрической модели многолучевого канала связи выводятся частные описания широкополосной SIMO (англ. Single Input Multiple Output - система с одной передающей и массивом приёмных антенн) системы и системы с отражателями в ближнем геометрическом поле.

\underline{\textbf{Вторая глава}} посвящена общим принципам, позволяющим оценивать параметры каналов связи с помощью алгоритмов обработки многомерных сигналов в широкополосных системах и системах с отражателями в ближнем геометрическом поле.

\underline{\textbf{Третья глава}} посвящена применению предложенных алгоритмов в конкретных инфокоммуникационных системах, описывает практические реализации и экспериментальные результаты.

В \underline{\textbf{заключении}} приведены основные результаты работы.