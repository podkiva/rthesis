\chapter{Общая характеристика работы}

\newcommand{\actuality}{\underline{\textbf{\actualityTXT}}}
\newcommand{\progress}{\underline{\textbf{\progressTXT}}}
\newcommand{\object}{\underline{{\textbf\objectTXT}}}
\newcommand{\predmet}{\underline{{\textbf\predmetTXT}}}
\newcommand{\aim}{\underline{{\textbf\aimTXT}}}
\newcommand{\tasks}{\underline{\textbf{\tasksTXT}}}
\newcommand{\novelty}{\underline{\textbf{\noveltyTXT}}}
\newcommand{\influence}{\underline{\textbf{\influenceTXT}}}
\newcommand{\methods}{\underline{\textbf{\methodsTXT}}}
\newcommand{\defpositions}{\underline{\textbf{\defpositionsTXT}}}
\newcommand{\reliability}{\underline{\textbf{\reliabilityTXT}}}
\newcommand{\probation}{\underline{\textbf{\probationTXT}}}
\newcommand{\contribution}{\underline{\textbf{\contributionTXT}}}
\newcommand{\publications}{\underline{\textbf{\publicationsTXT}}}
\newcommand{\passconf}{\underline{\textbf{\passconfTXT}}}


{\actuality} Развитие новых методов модуляции с множеством поднесущих и систем передачи данных с множеством входных и выходных портов (MIMO) увеличивают количество получаемой информации и приводят к тому, что сигналы в таких системах могут быть описаны как многомерные массивы данных - тензоры, представляющие собой дискретизацию многомерных непрерывных сигналов.

Тензорное представление сигналов и моделей каналов связи открывает новые возможности к оцениванию параметров таких сигналов с помощью тензорных разложений. Такие тензорные разложения обладают свойствами уникальности и идентифицируемости, которые необходимы для корректного извлечения полезной информации из принимаемых сигналов и, помимо этого, могут сами нести полезную информацию в определённых сценариях.

Вместе с тем, увеличение объёма получаемой информации приводит к увеличению требований к вычислительным ресурсам таких систем. Моделирование передаточных функций каналов связи как полностью случайных и неизвестных стохастических величин приводит к нереализуемым методам и алгоритмам. Поэтому, в данной работе рассматриваются параметрические модели каналов связи \cite{Richter2005}, позволяющие уменьшить количество свободных параметров каналов связи тем самым снизив требования к вычислительным ресурсам систем связи, а также увеличив эффективность их эквализации.

Новые инфокоммуникационные системы, в погоне за растущими потребностями в пропускной способности, постоянно наращивают используемые ресурсы, чаще всего за счёт использования большей полосы частот. Увеличение относительных полос частот этих систем приводит к несостоятельности традиционных моделей данных в них, и, как следствие, алгоритмов оценивания параметров \cite{DoHong2004, Raimondi2016}. В данной работе предлагается общая методика обработки получаемых данных, позволяющая применять узкополосные алгоритмы оценивания параметров каналов связи - направлений прихода сигналов на массив антенн - в широкополосных системах.

Сферическая модель фронта волны даёт возможность оценивать с помощью массива антенн не только направления прихода сигналов, но и расстояния до источника сигнала или его последнего отражения \cite{Singh2017a, Singh2017}. Получение с помощью тензорных разложений несмещенных оценок фазовых сигнатур приходящих на массив антенн сигналов позволяет применять новые методы определения местоположения источников сигналов и их отражателей в ближнем геометрическом поле \cite{Singh2016}. В данной работе предлагается новый алгоритм оценивания направлений прихода сигналов и расстояния до его источника в ближнем геометрическом поле.

Растущая насыщенность частотного спектра приводит к усложнению внутренних и внешних помех в современных инфокоммуникационных системах, что, в свою очередь, приводит к несостоятельности простых статистических моделей аддитивных помех в виде часто используемого белого шума с нормальным распределением \cite{Kozick2000, Kalyani2012}. Это оправдывает использование более сложных статистических моделей аддитивных помех в таких системах, например, таких как смесей нормальных распределений. Введение более сложных моделей аддитивных помех приводит к необходимости анализа потенциальных характеристик оценивания параметров сигналов, в качестве которых распространено использование нижней границы Крамера-Рао \cite{Cramer1993, KayS.1993}. В данной работе предлагается обобщенный метод расчёта потенциальных характеристик оценивания параметров каналов связи с учётом негауссовских распределений аддитивных помех.

{\object} беспроводные инфокоммуникационные системы с множеством приёмных и/или передающих антенн.

{\predmet} алгоритмы оценивания параметров многомерных сигналов в беспроводных инфокоммуникационных системах.

\begin{comment}
 Обзор, введение в тему, обозначение места данной работы в
мировых исследованиях и~т.\:п., можно использовать ссылки на~другие
работы~\autocite{Gosele1999161}
(если их~нет, то~в~автореферате
автоматически пропадёт раздел <<Список литературы>>). Внимание! Ссылки
на~другие работы в~разделе общей характеристики работы можно
использовать только при использовании \verb!biblatex! (из-за технических
ограничений \verb!bibtex8!. Это связано с тем, что одна
и~та~же~характеристика используются и~в~тексте диссертации, и в
автореферате. В~последнем, согласно ГОСТ, должен присутствовать список
работ автора по~теме диссертации, а~\verb!bibtex8! не~умеет выводить в~одном
файле два списка литературы).
При использовании \verb!biblatex! возможно использование исключительно
в~автореферате подстрочных ссылок
для других работ командой \verb!\autocite!, а~также цитирование
собственных работ командой \verb!\cite!. Для этого в~файле
\verb!common/setup.tex! необходимо присвоить положительное значение
счётчику \verb!\setcounter{usefootcite}{1}!.

Для генерации содержимого титульного листа автореферата, диссертации
и~презентации используются данные из файла \verb!common/data.tex!. Если,
например, вы меняете название диссертации, то оно автоматически
появится в~итоговых файлах после очередного запуска \LaTeX. Согласно
ГОСТ 7.0.11-2011 <<5.1.1 Титульный лист является первой страницей
диссертации, служит источником информации, необходимой для обработки и
поиска документа>>. Наличие логотипа организации на~титульном листе
упрощает обработку и~поиск, для этого разметите логотип вашей
организации в папке images в~формате PDF (лучше найти его в векторном
варианте, чтобы он хорошо смотрелся при печати) под именем
\verb!logo.pdf!. Настроить размер изображения с логотипом можно
в~соответствующих местах файлов \verb!title.tex!  отдельно для
диссертации и автореферата. Если вам логотип не~нужен, то просто
удалите файл с~логотипом.

\ifsynopsis
Этот абзац появляется только в~автореферате.
Для формирования блоков, которые будут обрабатываться только в~автореферате,
заведена проверка условия \verb!\!\verb!ifsynopsis!.
Значение условия задаётся в~основном файле документа (\verb!synopsis.tex! для
автореферата).
\else
Этот абзац появляется только в~диссертации.
Через проверку условия \verb!\!\verb!ifsynopsis!, задаваемого в~основном файле
документа (\verb!dissertation.tex! для диссертации), можно сделать новую
команду, обеспечивающую появление цитаты в~диссертации, но~не~в~автореферате.
\fi

При использовании пакета \verb!biblatex! будут подсчитаны все работы, добавленные
в файл \verb!biblio/author.bib!. Для правильного подсчёта работ в~различных
системах цитирования требуется использовать поля:
\begin{itemize}
\item \texttt{authorvak} если публикация индексирована ВАК,
\item \texttt{authorscopus} если публикация индексирована Scopus,
\item \texttt{authorwos} если публикация индексирована Web of Science,
\item \texttt{authorconf} для докладов конференций,
\item \texttt{authorother} для других публикаций.
\end{itemize}
Для подсчёта используются счётчики:
\begin{itemize}
\item \texttt{citeauthorvak} для работ, индексируемых ВАК,
\item \texttt{citeauthorscopus} для работ, индексируемых Scopus,
\item \texttt{citeauthorwos} для работ, индексируемых Web of Science,
\item \texttt{citeauthorvakscopuswos} для работ, индексируемых одной из трёх баз,
\item \texttt{citeauthorscopuswos} для работ, индексируемых Scopus или Web of~Science,
\item \texttt{citeauthorconf} для докладов на конференциях,
\item \texttt{citeauthorother} для остальных работ,
\item \texttt{citeauthor} для суммарного количества работ.
\end{itemize}
% Счётчик \texttt{citeexternal} используется для подсчёта процитированных публикаций.

Для добавления в список публикаций автора работ, которые не были процитированы в
автореферате требуется их~перечислить с использованием команды \verb!\nocite! в
\verb!Synopsis/content.tex!.

\end{comment}

{\progress} В работе Richter A. \cite{Richter2005}, в которой даётся крайне общее описание параметрической модели радиочастотного канала связи, учитывающей все основные физические эффекты, влияющие на распространение электро-магнитных волн. Однако в данной работе не развивается представление передаточной функции канала связи как многомерного массива данных - тензора, которое ведёт к новым подходам, методам и алгоритмам обработки и оценивания параметров каналов связи.

Описание моделей широкополосных каналов связи и соответствующих алгоритмов оценивания их параметров можно найти в таких работах как \cite{DDW93, HK90, OK90, VB88, WK85, KS90, FW93, CM89, KV96, AeroSK94, AeroCC93}.

Работы по локализации источников сигналов в ближнем геометрическом поле можно найти в \cite{Singh2016, Singh2017a, Singh2017}.

Исследования и алгоритмы оценивания для систем с негауссовским распределением аддитивных помех можно найти в \cite{Kozick2000, Kalyani2012}.

{\aim} данной работы является повышение эффективности инфокоммуникационных систем путём разработки новых алгоритмов оценивания параметров каналов связи, в том числе для широкополосных каналов связи, каналов связи с отражателями в ближнем геометрическом поле, а также для каналов связи с негауссовым распределением аддитивных помех.

Для~достижения поставленной цели необходимо было решить следующие {\tasks}:
\begin{enumerate}
  \item Исследовать и систематизировать параметрические модели каналов связи, в том числе модели широкополосных каналов связи, модели каналов связи с отражателями в ближнем геометрическом поле и модели каналов связи с негауссовым распределением аддитивных помех.
  \item Разработать методику обработки принятых сигналов для широкополосных каналов связи.
  \item Разработать алгоритм оценивания параметров канала связи с отражателями в ближнем геометрическом поле с использованием сферической модели фронта волны.
  \item Разработать метод вычисления границы Крамера-Рао оценивания параметров каналов связи с негауссовым распределением аддитивной помехи.
\end{enumerate}


{\novelty}
\begin{enumerate}
  \item Впервые предложен метод предварительной обработки многомерных сигналов в широкополосных системах, позволяющий применять алгоритмы оценивания параметров каналов связи, разработанные для узкополосных систем.
  \item Разработан новый алгоритм оценивания параметров канала связи с отражателями в ближнем геометрическом поле с использованием сферической модели фронта волны.
  \item Впервые исследована граница Крамера-Рао для задач оценки параметров каналов связи с негауссовым распределением аддитивной помехи, заданным смесью нормальных распределений с ненулевыми средними значениями компонент.
\end{enumerate}

{\influence} Теоретическая значимость работы состоит в следующем:
\begin{itemize}
	\item доказана эффективность метода предварительной обработки многомерных сигналов в широкополосных системах с относительной полосой частот, превышающей 10\%;
	\item показано, что разработанный алгоритм оценивания параметров канала связи с отражателями в ближнем геометрическом поле более эффективен чем существующие алгоритмы;
	\item показано, что использование более сложных моделей аддитивных помех потенциально позволяет увеличить точность работы алгоритмов оценивания параметров каналов связи.
\end{itemize}

Практическая значимость работы состоит в следующем:
\begin{itemize}
	\item программная реализация предложенных алгоритмов;
	\item разработаны компьютерные модели, имитирующие работу инфокоммуникационных систем с предложенными алгоритмами и оценивающие эффективность их работы.	
\end{itemize}

{\methods} При решении поставленных задач научного исследования использовались алгоритмы тензорных разложений, методы линейной алгебры, теория оценивания, теория вероятностей и статистики, методы обработки цифровых сигналов, методы компьютерного моделирования (в частности, метод Монте-Карло) и экспериментального исследования. \ldots

{\defpositions}
\begin{enumerate}
  \item Предложенный метод предварительной обработки многомерных сигналов в широкополосных системах существенно повышает точность алгоритмов оценки параметров каналов связи, разработанных для узкополосных систем. Улучшение точности оценивания увеличивается при увеличении относительной полосы частот рассматриваемой широкополосной системы.
  \item Разработанные алгоритм оценивания параметров канала связи с отражателями в ближнем геометрическом поле обеспечивает лучшую точность оценивания в сравнение с существующими методами.
  \item Исследование границы Крамера-Рао для систем с негауссовским распределением аддитивной помехи, выраженным смесью нормальных распределений, показало потенциальный задел на увеличение эффективности алгоритмов оценивания параметров каналов связи в таких системах.
\end{enumerate}

{\reliability} Достоверность результатов, полученных в ходе данной работы, подтверждается соответствием результатов теоретического анализа  результатам имитационного моделирования, а также результатам других авторов.

{\probation}

{\contribution} Все результаты, приведённые в основных положениях, выносимых на защиту, получены автором самостоятельно. Из работ, опубликованных в соавторстве, в диссертацию включена та их часть, которая получена автором лично.

\begin{comment}
{\passconf} Содержание диссертации соответствует паспорту научной специальности \thesisSpecialtyTwoNumber – \thesisSpecialtyTwoTitle, по пунктам:
\begin{description}
	\item[П.2] Исследование процессов генерации, представления, передачи, хранения и отображения аналоговой, цифровой, видео-, аудио- и мультимедиа информации; разработка рекомендаций по совершенствованию и созданию новых соответствующих алгоритмов и процедур;
	\item[П.8] Исследование и разработка новых сигналов, модемов, кодеков, мультиплексоров и селекторов, обеспечивающих высокую надёжность обмена информацией в условиях воздействия внешних и внутренних помех;
	\item[П.11] Разработка научно-технических основ технологии создания сетей, систем и устройств телекоммуникаций и обеспечения их эффективного функционирования.
\end{description}
\end{comment}

\ifnumequal{\value{bibliosel}}{0}
{%%% Встроенная реализация с загрузкой файла через движок bibtex8. (При желании, внутри можно использовать обычные ссылки, наподобие `\cite{vakbib1,vakbib2}`).
    {\publications} Основные результаты по теме диссертации изложены
    в~XX~печатных изданиях,
    X из которых изданы в журналах, рекомендованных ВАК,
    X "--- в тезисах докладов.
}%
{%%% Реализация пакетом biblatex через движок biber
    \begin{refsection}[bl-author]
        % Это refsection=1.
        % Процитированные здесь работы:
        %  * подсчитываются, для автоматического составления фразы "Основные результаты ..."
        %  * попадают в авторскую библиографию, при usefootcite==0 и стиле `\insertbiblioauthor` или `\insertbiblioauthorgrouped`
        %  * нумеруются там в зависимости от порядка команд `\printbibliography` в этом разделе.
        %  * при использовании `\insertbiblioauthorgrouped`, порядок команд `\printbibliography` в нём должен быть тем же (см. biblio/biblatex.tex)
        %
        % Невидимый библиографический список для подсчёта количества публикаций:
        \printbibliography[heading=nobibheading, section=1, env=countauthorvak,          keyword=biblioauthorvak]%
        \printbibliography[heading=nobibheading, section=1, env=countauthorwos,          keyword=biblioauthorwos]%
        \printbibliography[heading=nobibheading, section=1, env=countauthorscopus,       keyword=biblioauthorscopus]%
        \printbibliography[heading=nobibheading, section=1, env=countauthorconf,         keyword=biblioauthorconf]%
        \printbibliography[heading=nobibheading, section=1, env=countauthorother,        keyword=biblioauthorother]%
        \printbibliography[heading=nobibheading, section=1, env=countauthor,             keyword=biblioauthor]%
        \printbibliography[heading=nobibheading, section=1, env=countauthorvakscopuswos, filter=vakscopuswos]%
        \printbibliography[heading=nobibheading, section=1, env=countauthorscopuswos,    filter=scopuswos]%
        %
        \nocite{*}%
        %
        {\publications} Основные результаты по теме диссертации изложены в~\arabic{citeauthor}~печатных изданиях,
        \arabic{citeauthorvak} из которых изданы в журналах, рекомендованных ВАК\sloppy%
        \ifnum \value{citeauthorscopuswos}>0%
            , \arabic{citeauthorscopuswos} "--- в~периодических научных журналах, индексируемых Web of~Science и Scopus\sloppy%
        \fi%
        \ifnum \value{citeauthorconf}>0%
            , \arabic{citeauthorconf} "--- в~тезисах докладов.
        \else%
            .
        \fi
    \end{refsection}%
    \begin{refsection}[bl-author]
        % Это refsection=2.
        % Процитированные здесь работы:
        %  * попадают в авторскую библиографию, при usefootcite==0 и стиле `\insertbiblioauthorimportant`.
        %  * ни на что не влияют в противном случае
        % \nocite{vakbib2}%vak
        % \nocite{otherbib1}%other
        % \nocite{ptitt18}%conf
    \end{refsection}%
        %
        % Всё, что вне этих двух refsection, это refsection=0,
        %  * для диссертации - это нормальные ссылки, попадающие в обычную библиографию
        %  * для автореферата:
        %     * при usefootcite==0, ссылка корректно сработает только для источника из `external.bib`. Для своих работ --- напечатает "[0]" (и даже Warning не вылезет).
        %     * при usefootcite==1, ссылка сработает нормально. В авторской библиографии будут только процитированные в refsection=0 работы.
        %
        % Невидимый библиографический список для подсчёта количества внешних публикаций
        % Используется, чтобы убрать приставку "А" у работ автора, если в автореферате нет
        % цитирований внешних источников.
        % Замедляет компиляцию
    \ifsynopsis
    \ifnumequal{\value{draft}}{0}{
      \printbibliography[heading=nobibheading, section=0, env=countexternal,          keyword=biblioexternal]%
    }{}
    \fi
}

 % Характеристика работы по структуре во введении и в автореферате не отличается (ГОСТ Р 7.0.11, пункты 5.3.1 и 9.2.1), потому её загружаем из одного и того же внешнего файла, предварительно задав форму выделения некоторым параметрам

%Диссертационная работа была выполнена при поддержке грантов \dots

%\underline{\textbf{Объем и структура работы.}} Диссертация состоит из~введения,
%четырех глав, заключения и~приложения. Полный объем диссертации
%\textbf{ХХХ}~страниц текста с~\textbf{ХХ}~рисунками и~5~таблицами. Список
%литературы содержит \textbf{ХХX}~наименование.

\chapter*{Введение}                         % Заголовок
\addcontentsline{toc}{chapter}{Введение}    % Добавляем его в оглавление

Во \underline{\textbf{введении}} обосновывается актуальность
исследований, проводимых в~рамках данной диссертационной работы,
приводится обзор научной литературы по~изучаемой проблеме,
формулируется цель, ставятся задачи работы, излагается научная новизна
и практическая значимость представляемой работы. 

\underline{\textbf{Первая глава}} описывает обобщённую теорию представления многомерных сигналов и каналов связи, их параметрическое представление. Затем, из общей параметрической модели многолучевого канала связи выводятся частные описания широкополосной SIMO (англ. Single Input Multiple Output - система с одной передающей и массивом приёмных антенн) системы и системы с отражателями в ближнем геометрическом поле.

\underline{\textbf{Вторая глава}} посвящена общим принципам, позволяющим оценивать параметры каналов связи с помощью алгоритмов обработки многомерных сигналов в широкополосных системах и системах с отражателями в ближнем геометрическом поле.

\underline{\textbf{Третья глава}} посвящена применению предложенных алгоритмов в конкретных инфокоммуникационных системах, описывает практические реализации и экспериментальные результаты.

В \underline{\textbf{заключении}} приведены основные результаты работы.    % Введение
\chapter{Оценивание параметров многомерных сигналов в инфокоммуникационных системах}\label{ch:ch1}

\section{Многомерные сигналы в инфокоммуникационных системах связи}\label{sec:ch1/sec1}

Большинство современных инфокоммуникационных систем имеют дискретную (цифровую) природу. Дискретизация во временной области, лежащая в основе цифровых систем связи, теперь стала столь же распространённой в частотной и пространственной областях.

Появление систем с множеством поднесущих - например, систем с Ортогональным Частотным разделением Каналов (ОЧРК, англ. Orthogonal Division Multiplexing - OFDM) и их различных модификаций - привнесло дискретизацию передаваемых сигналов в частотной области, а развитие систем MIMO (системы с множеством входных и выходных портов, англ. Multiple Input Multiple Output - MIMO) привнесло дискретизацию в пространстве.

Таким образом, принимаемые/передаваемые сигналы $s(\ta,\fra,\spa)$ в современных инфокоммуникационных системах представляются как сложные многомерные функции многих переменных - времени $\ta$, частоты $\fra$ и положения в пространстве $\spa$.

Современные приёмные и передающие устройства систем связи позволяют измерять/формировать волновые процессы в конечном числе точек этого многомерного измеряемого пространства:
\begin{equation}
\label{eq:ch1:1}
s(\ti,\fri,\spi)=s(\ta,\fra,\spa)\at[\Bigg]{\substack{\ta=T\ti \\ f=\scs\fri \\ \spa=\bm{p}_{\spi} }}\in\compl
\end{equation}
где $\ti$, $\fri$, $\spi$ - индексы отсчётов по времени, частоте и пространству, соответственно; $\ts$ - период дискретизации во времени; $\scs$ - период дискретизации по частоте (расстояние между поднесущими); $\bm{p}_{\spi}$ - точки пространства, в которых находятся приёмные/передающие устройства).

Уравнение \eqref{eq:ch1:1} подразумевает равномерную (с одинаковым шагом) дискретизацию сигналов во временной и частотной области, рассматриваемую в данной работе.

Такие возможности измерения волновых процессов (электро-магнитных, акустических, гидроакустических, сейсмических и т.д.) одновременно в пространстве, времени и частоте приводят к необходимости формулирования новых математических моделей представления используемых сигналов.

Как видно из уравнения~\eqref{eq:ch1:1}, измеренные значения имеют трёх-мерную структуру, после дискретизации проще всего представляемую 3-х мерным массивом данных, или трёх-мерным тензором $\ten{S}$:
\begin{equation}
\label{eq:ch1:2}
[\ten{S}]_{\ti,\fri,\spi}=s(\ti,\fri,\spi)\in\compl^{\Nt\times \Nf \times \Ns}
\end{equation}
где $\Nt$, $\Nf$, $\Ns$ - количество отсчётов сигнала по времени, частоте и пространству, соответственно.

Стоит заметить, что излагаемые представления в равной степени относятся и к системам, работающим с другими типами волновых процессов - например, с акустическими, гидроакустическими, сейсмическими и т.д. Это означает, что разрабатываемые алгоритмы также применимы и в этих других областях техники.

Тензорное представление исследуемых сигналов позволяет не только упростить запись математических операций, но и легче выявлять скрытую структуру данных, позволяющую применять тензорные разложения (например, такие как \fixme{ссылки на тензорные разложения}) для оценки скрытых параметров моделей. Тензорные разложения, в отличие от их эквивалентов для матриц (например, \fixme{SVD, EVD,...}), обладают свойствами уникальности и идентифицируемости при выполнении более мягких (например, в сравнении с матрицами - \fixme{SVD не уникально}) требований.

\section{Параметрические каналы связи.}\label{sec:ch1/sec2}

Подробное теоретическое описание параметрических каналов можно найти в таких работах как \cite{Richter2005}. Приведём здесь краткое изложение модели параметрического канала связи, используемой в данной работе.

Любая инфокоммуникационная система обладает ещё большим количеством степеней свободы, так как рассматривается система с множеством выходных (приёмных) и входных (передающих) портов. Общепринятым методом описания процесса распространения сигнала от передатчика к приёмнику является описание через передаточную характеристику канала связи $\ten{H}$. В системах MIMO, в общем виде, помимо времени и частоты, передаточная функция указывается для всех комбинаций входных и выходных портом (антенн), иными словами, может быть представлена как минимум 4-х мерным тензором.

Например, в общем виде отсчёты принятых сигналов систем MIMO можно записать как:
\begin{equation}
\label{eq:ch1:3}
y(\ti,\fri,\rxi) = \sum_{\txi=1}^{\Ntx} h(\ti,\fri,\rxi,\txi)\cdot s(\ti,\fri,\txi) + n(\ti,\fri,\rxi)
\end{equation}
где $0\le\rxi<\Nrx$ и $0\le\txi<\Ntx$ - индексы передающих и приёмных портов (антенн), соответственно; $\Ntx$, $\Nrx$ - количество передающих и приёмных портов (антенн); $s(\ti,\fri,\txi)$ - переданные сигналы с каждой антенны; $n(\ti,\fri,\rxi)$ - аддитивная помеха/шум.

Таким образом, передаточная характеристика канала может быть также представлена в виде тензора:
\begin{equation}
\label{eq:ch1:4}
[\ten{H}]_{\ti,\fri,\rxi,\txi}=h(\ti,\fri,\rxi,\txi)\in\compl^{\Nt\times \Nf \times \Nrx \times \Ntx}
\end{equation}

В частных случаях расположений передающих/приёмных портов, передаточная характеристика может иметь больше измерений. Например, при использовании прямоугольных антенных массивов (англ. Uniform Rectangular Array - URA), пространственные измерения могут быть дополнительно разбиты на измерения соответствующие осям плоскости массивов антенн.

Передаточная характеристика канала играет ключевую роль при восстановлении (оценивании) переданного сигнала $s()$ на стороне приёмника по искажённому принятому сигналу $y()$, поэтому одной из основных задач в любой инфокоммуникационной системе является \textit{эквализация} канала связи - т.е. удаление его влияния на переданный сигнал. 

В простейшем случае канал связи может быть оценён в процессе \textit{измерения канала связи} - т.е. когда переданный сигнал известен на приёмной стороне и используется для оценки канала. Однако, как видно из уравнения \eqref{eq:ch1:4} количество неизвестных $N=\Nt\Nf\Nrx\Ntx$ геометрически возрастает с увеличением количества используемых системой связи ресурсов (полосы частот, приёмных или передающих антенн), поэтому представление канала как полностью неизвестной величины является неэффективным решением.

Поэтому широкое распространение получили модели канала связи, использующие внутреннюю структуру канала для экономного его описания с помощью гораздо меньшего количества скрытых параметров. Такие каналы можно назвать параметрическими, а вектор неизвестных скрытых параметров в данном случае обозначается как $\pars\in\real^\Npars$ и содержит конечное количество вещественных параметров $\Npars$.

\subsection{Многолучевая модель канала связи}\label{subsec:ch1/sec1/sub1}

Самым распространённым подходом к моделированию канала связи является использование многолучевой модели распространения сигналов. В этом случае канал связи на каждой поднесущей $\fri$ и в каждый момент времени $\ti$ представляется как сумма конечного количества лучей, идущих от каждой передающей антенны к каждой приёмной.

Для использования такого описания канала связи примем следующие предположения:
\begin{assumption}
\label{as:ch1:1}
На каждой поднесущей $\fri$ и в каждый момент времени $\ti$, канал связи представляется как суперпозиция узкополосных каналов передачи (лучей) с постоянной частотной передаточной характеристикой.
\end{assumption}

Канал с постоянной частотной передаточной характеристикой во временной области имеет импульсную характеристику в виде смещённой дельта-функции $g(t)=A\sigma (t-\tau)$, что, в свою очередь эквивалентно описанию передаточной характеристики с помощью одного комплексного коэффициента $h=Ae^{-j\phi}$. 

Определим также множество $\set{L}$ индексов всех лучей канала связи от каждой передающей антенны $n$ к массиву приёмных антенн, а множество индексов всех лучей, идущих от $\txi$-й антенны обозначим как $\set{L}_n$. Следовательно, можно записать:
\begin{equation}
\label{eq:ch1:5}
\set{L} = \set{L}_0 \cup ... \cup \set{L}_{\Ntx-1}
\end{equation}

Таким образом, предположение~\ref{as:ch1:1} позволяет описывать канал связи всей системы как:
\begin{equation}
\label{eq:ch1:6}
h(\ti,\fri,\rxi,\txi)=\sum_{\pai\in\Spai} \hpa(\ti,\fri,\rxi,\txi)
\end{equation}
где $\pai\in\Spai$ - индекс пути распространения (луча) от передающей антенны $\txi$ к массиву приёмных антенн. Обратите внимание,  что в общем виде каждая антенна имеет $\Npa=|\Spai|$ независимых дискретных путей к массиву приёмных антенн. В частных случаях эти пути могут практически совпадать друг с другом, например, если передающие антенны находятся близко друг к другу (массив передающих антенн).
	
Коэффициент передачи одного пути $\hpa$ можно записать следующим образом:
\begin{equation}
\label{eq:ch1:7}
\hpa(\ti,\fri,\rxi,\txi) = \tgaini^{\ti,\fri,\rxi} \cdot 
                           e^{-j2\pi \sfr \fulldelay} \cdot 
                           e^{j2\pi \frac{\sfr \ts\ti}{c} \vrel^\rxi}
\end{equation}
где:
\begin{description}
	\item[$c$] - скорость распространения волны в среде (например, скорость света);
	\item[$\lfr$] - нижняя граничная частота системы (наименьшая частота поднесущей);
	\item[$\sfr=\lfr + \scs \fri$] - частота поднесущей;
	\item[$\tgaini^{\ti,\fri,\rxi}$] - комплексный коэффициент затухания сигнала по пути $\pai$ на $\fri$-й частоте в $\ti$-й момент времени к $\rxi$-й антенне, включающий эффекты затухания и сдвига фазы в среде распространения и при отражениях/рассеиваниях;
	\item[$\vrel^\rxi$] - относительная скорость движения по пути $\pai$ к антенне $\rxi$;
	\item[$\fulldelay$] - полная задержка распространения от антенны $\txi$ по пути $\pai$ к антенне $\rxi$.
\end{description}

Как видно из уравнения \eqref{eq:ch1:7}, передаточный коэффициент $h_\pai$ представляет несколько физических эффектов, влияющих на распространение сигналов:
\begin{enumerate}
	\item Затухания в среде и отражения/рассеивания от границ разделов разных сред
	\item Сдвиг фазы волны в результате распространения по пути луча
	\item Сдвиг фазы волны в результате эффекта Допплера (смещение Допплера), появляющийся в результате движения передающих и приёмных антенн друг относительно друга
\end{enumerate}

Для упрощения модели данных в системах MIMO справедливы следующие предположения:
\begin{assumption}
	\label{as:ch1:2}
	Относительная скорость вдоль любого луча любой передающей антенны одинакова для всех приёмных антенн, т.е. $\vrel^\rxi=\vrel$ $\forall$ $0\leq\rxi<\Nrx$.
\end{assumption}
\begin{assumption}
\label{as:ch1:3}
Коэффициент затухания сигнала одинаков для всех приёмных антенн, т.е. $\tgaini^{\ti,\fri,\rxi}=\tgaini^{\ti,\fri}$ $\forall$ $0\leq\rxi<\Nrx$.
\end{assumption}
\begin{assumption}
	\label{as:ch1:4}
	Коэффициент затухания сигнала не зависит от частоты, т.е. $\tgaini^{\ti,\fri}=\tgaini^{\ti}$ $\forall$ $0\leq\fri<\Nf$.
\end{assumption}

Предположение~\ref{as:ch1:2} говорит о том, что относительная скорость вдоль луча $\vrel$ не зависит от индекса приёмной антенны $\rxi$.

Предположения~\ref{as:ch1:2} и \ref{as:ch1:3} справедливы в случае когда приёмные антенны расположены рядом друг с другом, и расстояние между ними много меньше длин лучей $\di$, что чаще всего соблюдается в MIMO системах.

Обратите внимание, что верхний индекс $\txi$ величины $\fulldelay$ необязательно должен равняться индексу $\txi$ в $\pai$, т.е. величина $\fulldelay$ может определять задержку антенны относительно пути для другой антенны. Например, величина $\tilde{\tau}_{l_0=0}^{2,2}$ обозначает задержку от передающей антенны $\txi=2$ по пути $l_0=0$ к приёмной антенне $\rxi=2$.

Тогда, задержку распространения $\fulldelay$ можно представить в виде:
\begin{equation}
\label{eq:ch1:8}
\fulldelay = \tilde{\tau}^{\txi,0}_{\pai} + \frac{\delrx}{c}
\end{equation}
или
\begin{equation}
\label{eq:ch1:9}
\fulldelay = \tilde{\tau}^{0,0}_{\pai} + \frac{\deltx + \delrx}{c}
\end{equation}
где:
\begin{description}
	\item[$d_{\pai}^\txi$] - расстояние между $\txi$-й передающей и опорной приёмной антеннами по пути луча $\pai$;	
	\item[$\tilde{\tau}^{\txi,0}_{\pai}=\delay^\txi=d_{\pai}^\txi/c$] - задержка распространения между $\txi$-й передающей и опорной приёмной антеннами по пути луча $\pai$;
	\item[$\di$] - расстояние между опорной передающей и опорной приёмной антеннами по пути луча $\pai$;
	\item[$\tilde{\tau}^{0,0}_{\pai}=\delay=\di/c$] - задержка распространения между опорной передающей и опорной приёмной антеннами по пути луча $\pai$;
	\item[$\deltx$] - геометрическая разность хода фронта волны от $\txi$-й передающей антенны до <<опорной>> передающей антенны (с индексом 0) по пути луча $\pai$;
	\item[$\delrx$] - геометрическая разность хода фронта волны от $\rxi$-й приёмной антенны до <<опорной>> приёмной антенны (с индексом 0) по пути луча $\pai$;
\end{description}

Таким образом, учитывая предположение~\ref{as:ch1:2} и используя уравнение \eqref{eq:ch1:8}, уравнение~\eqref{eq:ch1:7} можно переписать в виде:
\begin{equation}
\begin{aligned}
\label{eq:ch1:10}
\hpa(\ti,\fri,\rxi,\txi) =&\underbrace{\tgaini^{\ti} \cdot e^{-j2\pi \lfr \delay^\txi}}_{\gain_{\pai,\txi}^\ti} \cdot 
						   \underbrace{e^{-j2\pi \scs \fri \delay^\txi}}_{b_{\pai,\fri,\txi}^f} \cdot
						   \underbrace{e^{j2\pi \frac{\sfr \ts\ti}{c} \vrel}}_{b_{\pai,\ti,\fri}^t} \cdot
						   \underbrace{e^{-j2\pi \frac{\sfr}{c} \delrx}}_{b_{\pai,\rxi,\fri}^\te{rx}} \\
						 =&\gain_{\pai,\txi}^\ti \cdot b_{\pai,\fri,\txi}^f \cdot b_{\pai,\ti,\fri}^t  \cdot b_{\pai,\rxi,\fri}^\te{rx}
\end{aligned}
\end{equation}
или используя уравнение \eqref{eq:ch1:9}:
\begin{equation}
\begin{aligned}
\label{eq:ch1:11}
\hpa(\ti,\fri,\rxi,\txi) =&\underbrace{\tgaini^{\ti} \cdot e^{-j2\pi \lfr \delay}}_\gaini \cdot 
						   \underbrace{e^{-j2\pi \scs \fri \delay}}_{b_{\pai,\fri}^f} \cdot
					       \underbrace{e^{j2\pi \frac{\sfr \ts\ti}{c} \vrel}}_{b_{\pai,\ti,\fri}^t} \cdot
						   \underbrace{e^{-j2\pi \frac{\sfr}{c} \delrx}}_{b_{\pai,\rxi,\fri}^\te{rx}} \cdot
						   \underbrace{e^{-j2\pi \frac{\sfr}{c} \deltx}}_{b_{\pai,\txi,\fri}^\te{tx}} \\
						 =&\gaini \cdot b_{\pai,\fri}^f \cdot b_{\pai,\ti,\fri}^t \cdot b_{\pai,\rxi,\fri}^\te{rx} \cdot b_{\pai,\txi,\fri}^\te{tx}
\end{aligned}
\end{equation}

Представление~\eqref{eq:ch1:10} подходит для систем с разнесёнными (не объединёнными в один массив) передатчиками. Например, если рассматривается восходящая линия мобильной сети, в которой каждая передающая антенна это один абонент сети.

Представление~\eqref{eq:ch1:11}, напротив, подходит для случая системы MIMO с массивом близкорасположенных передающих антенн. В этом случае удобнее выразить задержки относительно одной <<опорной>> передающей антенны.

\begin{comment}
% \cite[с.~54]{Sokolov}\cite[с.~36]{Gaidaenko}.
% Две ссылки: \cite{Sokolov,Gaidaenko}.
% Ссылка на собственные работы: \cite{vakbib1, confbib2}.
% Много ссылок: %\cite[с.~54]{Lermontov,Management,Borozda} % такой «фокус»
%вызывает biblatex warning относительно опции sortcites, потому что неясно, к
%какому источнику относится уточнение о страницах, а bibtex об этой проблеме
%даже не предупреждает
% \cite{Lermontov, Management, Borozda, Marketing, Constitution, FamilyCode,
% Gost.7.0.53, Razumovski, Lagkueva, Pokrovski, Methodology, Nasirova, Berestova,
% Kriger}%
% \ifnumequal{\value{bibliosel}}{0}{% Примеры для bibtex8
%     \cite{Sirotko, Lukina, Encyclopedia}%
% }{% Примеры для biblatex через движок biber
%     \cite{Sirotko2, Lukina2, Encyclopedia2}%
% }%


\section{Формулы}\label{sec:ch1/sec4}

Благодаря пакету \textit{icomma}, \LaTeX~одинаково хорошо воспринимает
в~качестве десятичного разделителя и запятую (\(3,1415\)), и точку (\(3.1415\)).

\subsection{Ненумерованные одиночные формулы}\label{subsec:ch1/sec4/sub1}

Вот так может выглядеть формула, которую необходимо вставить в~строку
по~тексту: \(x \approx \sin x\) при \(x \to 0\).

А вот так выглядит ненумерованная отдельностоящая формула c подстрочными
и надстрочными индексами:
\[
(x_1+x_2)^2 = x_1^2 + 2 x_1 x_2 + x_2^2
\]

Формула с неопределенным интегралом:
\[
\int f(\alpha+x)=\sum\beta
\]

При использовании дробей формулы могут получаться очень высокие:
\[
  \frac{1}{\sqrt{2}+
  \displaystyle\frac{1}{\sqrt{2}+
  \displaystyle\frac{1}{\sqrt{2}+\cdots}}}
\]

В формулах можно использовать греческие буквы:
%Все \original... команды заранее, ради этого примера, определены в Dissertation\userstyles.tex
\[
\alpha\beta\gamma\delta\originalepsilon\epsilon\zeta\eta\theta%
\vartheta\iota\kappa\varkappa\lambda\mu\nu\xi\pi\varpi\rho\varrho%
\sigma\varsigma\tau\upsilon\originalphi\phi\chi\psi\omega\Gamma\Delta%
\Theta\Lambda\Xi\Pi\Sigma\Upsilon\Phi\Psi\Omega
\]
\[%https://texfaq.org/FAQ-boldgreek
\boldsymbol{\alpha\beta\gamma\delta\originalepsilon\epsilon\zeta\eta%
\theta\vartheta\iota\kappa\varkappa\lambda\mu\nu\xi\pi\varpi\rho%
\varrho\sigma\varsigma\tau\upsilon\originalphi\phi\chi\psi\omega\Gamma%
\Delta\Theta\Lambda\Xi\Pi\Sigma\Upsilon\Phi\Psi\Omega}
\]

Для добавления формул можно использовать пары \verb+$+\dots\verb+$+ и \verb+$$+\dots\verb+$$+,
но~они считаются устаревшими.
Лучше использовать их функциональные аналоги \verb+\(+\dots\verb+\)+ и \verb+\[+\dots\verb+\]+.

\subsection{Ненумерованные многострочные формулы}\label{subsec:ch1/sec4/sub2}

Вот так можно написать две формулы, не нумеруя их, чтобы знаки <<равно>> были
строго друг под другом:
\begin{align}
  f_W & =  \min \left( 1, \max \left( 0, \frac{W_{soil} / W_{max}}{W_{crit}} \right)  \right), \nonumber \\
  f_T & =  \min \left( 1, \max \left( 0, \frac{T_s / T_{melt}}{T_{crit}} \right)  \right), \nonumber
\end{align}

Выровнять систему ещё и по переменной \( x \) можно, используя окружение
\verb|alignedat| из пакета \verb|amsmath|. Вот так:
\[
    |x| = \left\{
    \begin{alignedat}{2}
        &&x, \quad &\text{eсли } x\geqslant 0 \\
        &-&x, \quad & \text{eсли } x<0
    \end{alignedat}
    \right.
\]
Здесь первый амперсанд (в исходном \LaTeX\ описании формулы) означает
выравнивание по~левому краю, второй "--- по~\( x \), а~третий "--- по~слову
<<если>>. Команда \verb|\quad| делает большой горизонтальный пробел.

Ещё вариант:
\[
    |x|=
    \begin{cases}
    \phantom{-}x, \text{если } x \geqslant 0 \\
    -x, \text{если } x<0
    \end{cases}
\]

Кроме того, для  нумерованных формул \verb|alignedat| делает вертикальное
выравнивание номера формулы по центру формулы. Например, выравнивание
компонент вектора:
\begin{equation}
\label{eq:2p3}
\begin{alignedat}{2}
{\mathbf{N}}_{o1n}^{(j)} = \,{\sin} \phi\,n\!\left(n+1\right)
         {\sin}\theta\,
         \pi_n\!\left({\cos} \theta\right)
         \frac{
               z_n^{(j)}\!\left( \rho \right)
              }{\rho}\,
           &{\boldsymbol{\hat{\mathrm e}}}_{r}\,+   \\
+\,
{\sin} \phi\,
         \tau_n\!\left({\cos} \theta\right)
         \frac{
            \left[\rho z_n^{(j)}\!\left( \rho \right)\right]^{\prime}
              }{\rho}\,
            &{\boldsymbol{\hat{\mathrm e}}}_{\theta}\,+   \\
+\,
{\cos} \phi\,
         \pi_n\!\left({\cos} \theta\right)
         \frac{
            \left[\rho z_n^{(j)}\!\left( \rho \right)\right]^{\prime}
              }{\rho}\,
            &{\boldsymbol{\hat{\mathrm e}}}_{\phi}\:.
\end{alignedat}
\end{equation}

Ещё об отступах. Иногда для лучшей <<читаемости>> формул полезно
немного исправить стандартные интервалы \LaTeX\ с учётом логической
структуры самой формулы. Например в формуле~\ref{eq:2p3} добавлен
небольшой отступ \verb+\,+ между основными сомножителями, ниже
результат применения всех вариантов отступа:
\begin{align*}
\backslash! &\quad f(x) = x^2\! +3x\! +2 \\
  \mbox{по-умолчанию} &\quad f(x) = x^2+3x+2 \\
\backslash, &\quad f(x) = x^2\, +3x\, +2 \\
\backslash{:} &\quad f(x) = x^2\: +3x\: +2 \\
\backslash; &\quad f(x) = x^2\; +3x\; +2 \\
\backslash \mbox{space} &\quad f(x) = x^2\ +3x\ +2 \\
\backslash \mbox{quad} &\quad f(x) = x^2\quad +3x\quad +2 \\
\backslash \mbox{qquad} &\quad f(x) = x^2\qquad +3x\qquad +2
\end{align*}

Можно использовать разные математические алфавиты:
\begin{align}
\mathcal{ABCDEFGHIJKLMNOPQRSTUVWXYZ} \nonumber \\
\mathfrak{ABCDEFGHIJKLMNOPQRSTUVWXYZ} \nonumber \\
\mathbb{ABCDEFGHIJKLMNOPQRSTUVWXYZ} \nonumber
\end{align}

Посмотрим на систему уравнений на примере аттрактора Лоренца:

\[
\left\{
  \begin{array}{rl}
    \dot x = & \sigma (y-x) \\
    \dot y = & x (r - z) - y \\
    \dot z = & xy - bz
  \end{array}
\right.
\]

А для вёрстки матриц удобно использовать многоточия:
\[
\left(
  \begin{array}{ccc}
    a_{11} & \ldots & a_{1n} \\
    \vdots & \ddots & \vdots \\
    a_{n1} & \ldots & a_{nn} \\
  \end{array}
\right)
\]

\subsection{Нумерованные формулы}\label{subsec:ch1/sec4/sub3}

А вот так пишется нумерованная формула:
\begin{equation}
  \label{eq:equation1}
  e = \lim_{n \to \infty} \left( 1+\frac{1}{n} \right) ^n
\end{equation}

Нумерованных формул может быть несколько:
\begin{equation}
  \label{eq:equation2}
  \lim_{n \to \infty} \sum_{k=1}^n \frac{1}{k^2} = \frac{\pi^2}{6}
\end{equation}

Впоследствии на формулы~\eqref{eq:equation1} и~\eqref{eq:equation2} можно ссылаться.

Сделать так, чтобы номер формулы стоял напротив средней строки, можно,
используя окружение \verb|multlined| (пакет \verb|mathtools|) вместо
\verb|multline| внутри окружения \verb|equation|. Вот так:
\begin{equation} % \tag{S} % tag - вписывает свой текст
  \label{eq:equation3}
    \begin{multlined}
        1+ 2+3+4+5+6+7+\dots + \\
        + 50+51+52+53+54+55+56+57 + \dots + \\
        + 96+97+98+99+100=5050
    \end{multlined}
\end{equation}

Используя команду \verb|\eqrefs|, можно
красиво ссылаться сразу на несколько формул
\eqrefs{eq:equation1, eq:equation3, eq:equation2}, даже перепутав
порядок ссылок \verb|\eqrefs{eq1, eq3, eq2}|.
Аналогично, для ссылок на несколько рисунков, таблиц и~т.\:д.
\refs{sec:ch1/sec1, sec:ch1/sec2, sec:ch1/sec3} можно использовать
команду \verb|\refs|.
Обе эти команды определены в файле \verb|common/packages.tex|.

Уравнения~\eqrefs{eq:subeq_1,eq:subeq_2} демонстрируют возможности
окружения \verb|\subequations|.
\begin{subequations}
    \label{eq:subeq_1}
    \begin{gather}
        y = x^2 + 1 \label{eq:subeq_1-1} \\
        y = 2 x^2 - x + 1 \label{eq:subeq_1-2}
    \end{gather}
\end{subequations}
Ссылки на отдельные уравнения~\eqrefs{eq:subeq_1-1,eq:subeq_1-2,eq:subeq_2-1}.
\begin{subequations}
    \label{eq:subeq_2}
    \begin{align}
        y &= x^3 + x^2 + x + 1 \label{eq:subeq_2-1} \\
        y &= x^2
    \end{align}
\end{subequations}

\subsection{Форматирование чисел и размерностей величин}\label{sec:units}

Числа форматируются при помощи команды \verb|\num|:
\num{5,3};
\num{2,3e8};
\num{12345,67890};
\num{2,6 d4};
\num{1+-2i};
\num{.3e45};
\num[exponent-base=2]{5 e64};
\num[exponent-base=2,exponent-to-prefix]{5 e64};
\num{1.654 x 2.34 x 3.430}
\num{1 2 x 3 / 4}.
Для написания последовательности чисел можно использовать команды \verb|\numlist| и \verb|\numrange|:
\numlist{10;30;50;70}; \numrange{10}{30}.
Значения углов можно форматировать при помощи команды \verb|\ang|:
\ang{2.67};
\ang{30,3};
\ang{-1;;};
\ang{;-2;};
\ang{;;-3};
\ang{300;10;1}.

Обратите внимание, что ГОСТ запрещает использование знака <<->> для обозначения отрицательных чисел
за исключением формул, таблиц и~рисунков.
Вместо него следует использовать слово <<минус>>.

Размерности можно записывать при помощи команд \verb|\si| и \verb|\SI|:
\si{\farad\squared\lumen\candela};
\si{\joule\per\mole\per\kelvin};
\si[per-mode = symbol-or-fraction]{\joule\per\mole\per\kelvin};
\si{\metre\per\second\squared};
\SI{0.10(5)}{\neper};
\SI{1.2-3i e5}{\joule\per\mole\per\kelvin};
\SIlist{1;2;3;4}{\tesla};
\SIrange{50}{100}{\volt}.
Список единиц измерений приведён в таблицах~\refs{tab:unit:base,
tab:unit:derived,tab:unit:accepted,tab:unit:physical,tab:unit:other}.
Приставки единиц приведены в~таблице~\ref{tab:unit:prefix}.

С дополнительными опциями форматирования можно ознакомиться в~описании пакета \texttt{siunitx};
изменить или добавить единицы измерений можно в~файле \texttt{siunitx.cfg}.

\begin{table}
    \centering
    \captionsetup{justification=centering} % выравнивание подписи по-центру
    \caption{Основные величины СИ}\label{tab:unit:base}
    \begin{tabular}{llc}
        \toprule
        Название  & Команда                & Символ         \\
        \midrule
        Ампер     & \verb|\ampere| & \si{\ampere}   \\
        Кандела   & \verb|\candela| & \si{\candela}  \\
        Кельвин   & \verb|\kelvin| & \si{\kelvin}   \\
        Килограмм & \verb|\kilogram| & \si{\kilogram} \\
        Метр      & \verb|\metre| & \si{\metre}    \\
        Моль      & \verb|\mole| & \si{\mole}     \\
        Секунда   & \verb|\second| & \si{\second}   \\
        \bottomrule
    \end{tabular}
\end{table}

\begin{table}
  \small
  \centering
  \begin{threeparttable}% выравнивание подписи по границам таблицы
    \caption{Производные единицы СИ}\label{tab:unit:derived}
    \begin{tabular}{llc|llc}
        \toprule
        Название       & Команда                 & Символ              & Название & Команда & Символ \\
        \midrule
        Беккерель      & \verb|\becquerel|  & \si{\becquerel}     &
        Ньютон         & \verb|\newton|  & \si{\newton}                                      \\
        Градус Цельсия & \verb|\degreeCelsius| & \si{\degreeCelsius} &
        Ом             & \verb|\ohm| & \si{\ohm}                                         \\
        Кулон          & \verb|\coulomb| & \si{\coulomb}       &
        Паскаль        & \verb|\pascal| & \si{\pascal}                                      \\
        Фарад          & \verb|\farad| & \si{\farad}         &
        Радиан         & \verb|\radian| & \si{\radian}                                      \\
        Грей           & \verb|\gray| & \si{\gray}          &
        Сименс         & \verb|\siemens| & \si{\siemens}                                     \\
        Герц           & \verb|\hertz| & \si{\hertz}         &
        Зиверт         & \verb|\sievert| & \si{\sievert}                                     \\
        Генри          & \verb|\henry| & \si{\henry}         &
        Стерадиан      & \verb|\steradian| & \si{\steradian}                                   \\
        Джоуль         & \verb|\joule| & \si{\joule}         &
        Тесла          & \verb|\tesla| & \si{\tesla}                                       \\
        Катал          & \verb|\katal| & \si{\katal}         &
        Вольт          & \verb|\volt| & \si{\volt}                                        \\
        Люмен          & \verb|\lumen| & \si{\lumen}         &
        Ватт           & \verb|\watt| & \si{\watt}                                        \\
        Люкс           & \verb|\lux| & \si{\lux}           &
        Вебер          & \verb|\weber| & \si{\weber}                                       \\
        \bottomrule
    \end{tabular}
  \end{threeparttable}
\end{table}

\begin{table}
  \centering
  \begin{threeparttable}% выравнивание подписи по границам таблицы
    \caption{Внесистемные единицы}\label{tab:unit:accepted}

    \begin{tabular}{llc}
        \toprule
        Название        & Команда                 & Символ          \\
        \midrule
        День            & \verb|\day| & \si{\day}       \\
        Градус          & \verb|\degree| & \si{\degree}    \\
        Гектар          & \verb|\hectare| & \si{\hectare}   \\
        Час             & \verb|\hour| & \si{\hour}      \\
        Литр            & \verb|\litre| & \si{\litre}     \\
        Угловая минута  & \verb|\arcminute| & \si{\arcminute} \\
        Угловая секунда & \verb|\arcsecond| & \si{\arcsecond} \\ %
        Минута          & \verb|\minute| & \si{\minute}    \\
        Тонна           & \verb|\tonne| & \si{\tonne}     \\
        \bottomrule
    \end{tabular}
  \end{threeparttable}
\end{table}

\begin{table}
    \centering
    \captionsetup{justification=centering}
    \caption{Внесистемные единицы, получаемые из эксперимента}\label{tab:unit:physical}
    \begin{tabular}{llc}
        \toprule
        Название                & Команда                 & Символ                 \\
        \midrule
        Астрономическая единица & \verb|\astronomicalunit| & \si{\astronomicalunit} \\
        Атомная единица массы   & \verb|\atomicmassunit| & \si{\atomicmassunit}   \\
        Боровский радиус        & \verb|\bohr| & \si{\bohr}             \\
        Скорость света          & \verb|\clight| & \si{\clight}           \\
        Дальтон                 & \verb|\dalton| & \si{\dalton}           \\
        Масса электрона         & \verb|\electronmass| & \si{\electronmass}     \\
        Электрон Вольт          & \verb|\electronvolt| & \si{\electronvolt}     \\
        Элементарный заряд      & \verb|\elementarycharge| & \si{\elementarycharge} \\
        Энергия Хартри          & \verb|\hartree| & \si{\hartree}          \\
        Постоянная Планка       & \verb|\planckbar| & \si{\planckbar}        \\
        \bottomrule
    \end{tabular}
\end{table}

\begin{table}
  \centering
  \begin{threeparttable}% выравнивание подписи по границам таблицы
    \caption{Другие внесистемные единицы}\label{tab:unit:other}
    \begin{tabular}{llc}
        \toprule
        Название                  & Команда                 & Символ             \\
        \midrule
        Ангстрем                  & \verb|\angstrom| & \si{\angstrom}     \\
        Бар                       & \verb|\bar| & \si{\bar}          \\
        Барн                      & \verb|\barn| & \si{\barn}         \\
        Бел                       & \verb|\bel| & \si{\bel}          \\
        Децибел                   & \verb|\decibel| & \si{\decibel}      \\
        Узел                      & \verb|\knot| & \si{\knot}         \\
        Миллиметр ртутного столба & \verb|\mmHg| & \si{\mmHg}         \\
        Морская миля              & \verb|\nauticalmile| & \si{\nauticalmile} \\
        Непер                     & \verb|\neper| & \si{\neper}        \\
        \bottomrule
    \end{tabular}
  \end{threeparttable}
\end{table}

\begin{table}
  \small
  \centering
  \begin{threeparttable}% выравнивание подписи по границам таблицы
    \caption{Приставки СИ}\label{tab:unit:prefix}
    \begin{tabular}{llcc|llcc}
        \toprule
        Приставка & Команда                 & Символ      & Степень &
        Приставка & Команда                 & Символ      & Степень   \\
        \midrule
        Иокто     & \verb|\yocto| & \si{\yocto} & -24     &
        Дека      & \verb|\deca| & \si{\deca}  & 1         \\
        Зепто     & \verb|\zepto| & \si{\zepto} & -21     &
        Гекто     & \verb|\hecto| & \si{\hecto} & 2         \\
        Атто      & \verb|\atto| & \si{\atto}  & -18     &
        Кило      & \verb|\kilo| & \si{\kilo}  & 3         \\
        Фемто     & \verb|\femto| & \si{\femto} & -15     &
        Мега      & \verb|\mega| & \si{\mega}  & 6         \\
        Пико      & \verb|\pico| & \si{\pico}  & -12     &
        Гига      & \verb|\giga| & \si{\giga}  & 9         \\
        Нано      & \verb|\nano| & \si{\nano}  & -9      &
        Терра     & \verb|\tera| & \si{\tera}  & 12        \\
        Микро     & \verb|\micro| & \si{\micro} & -6      &
        Пета      & \verb|\peta| & \si{\peta}  & 15        \\
        Милли     & \verb|\milli| & \si{\milli} & -3      &
        Екса      & \verb|\exa| & \si{\exa}   & 18        \\
        Санти     & \verb|\centi| & \si{\centi} & -2      &
        Зетта     & \verb|\zetta| & \si{\zetta} & 21        \\
        Деци      & \verb|\deci| & \si{\deci}  & -1      &
        Иотта     & \verb|\yotta| & \si{\yotta} & 24        \\
        \bottomrule
    \end{tabular}
  \end{threeparttable}
\end{table}

\subsection{Заголовки с формулами: \texorpdfstring{\(a^2 + b^2 = c^2\)}{%
a\texttwosuperior\ + b\texttwosuperior\ = c\texttwosuperior},
\texorpdfstring{\(\left\vert\textrm{{Im}}\Sigma\left(
\protect\varepsilon\right)\right\vert\approx const\)}{|ImΣ (ε)| ≈ const},
\texorpdfstring{\(\sigma_{xx}^{(1)}\)}{σ\_\{xx\}\textasciicircum\{(1)\}}
}\label{subsec:with_math}

Пакет \texttt{hyperref} берёт текст для закладок в pdf-файле из~аргументов
команд типа \verb|\section|, которые могут содержать математические формулы,
а~также изменения цвета текста или шрифта, которые не отображаются в~закладках.
Чтобы использование формул в заголовках не вызывало в~логе компиляции появление
предупреждений типа <<\texttt{Token not allowed in~a~PDF string
(Unicode):(hyperref) removing...}>>, следует использовать конструкцию
\verb|\texorpdfstring{}{}|, где в~первых фигурных скобках указывается
формула, а~во~вторых "--- запись формулы для закладок.

\section{Рецензирование текста}\label{sec:markup}

В шаблоне для диссертации и автореферата заданы команды рецензирования.
Они видны при компиляции шаблона в режиме черновика или при установке
соответствующей настройки (\verb+showmarkup+) в~файле \verb+common/setup.tex+.

Команда \verb+\todo+ отмечает текст красным цветом.
\todo{Например, так.}

Команда \verb+\note+ позволяет выбрать цвет текста.
\note{Чёрный, } \note[red]{красный, } \note[green]{зелёный, }
\note[blue]{синий.} \note[orange]{Обратите внимание на ширину и расстановку
формирующихся пробелов, в~результате приведённой записи (зависит также
от~применяемого компилятора).}

Окружение \verb+commentbox+ также позволяет выбрать цвет.

\begin{commentbox}[red]
        Красный текст.

        Несколько параграфов красного текста.
\end{commentbox}

\begin{commentbox}[blue]
        Синяя формула.

        \begin{equation}
                \alpha + \beta = \gamma
        \end{equation}
\end{commentbox}

\verb+commentbox+ позволяет закомментировать участок кода в~режиме чистовика.
Чтобы убрать кусок кода для всех режимов, можно использовать окружение
\verb+comment+.

        Этот текст всегда скрыт.
\end{comment}
     % Глава 1
\chapter{Оценивание параметров каналов связи с расширенными моделями}\label{ch:ch2}

\section{Оценивание параметров каналов связи в широкополосных системах}\label{ch:ch2/sec1}

Результаты данного раздела доклада были опубликованы в \cite{Zhang2017, Podkurkov2019}.

Рассмотрим систему с Ортогональ




Это подход с интерполяцией входных данных. История, первые источники, что сделано нами, применение в многомерном пространстве, анализ характеристик.

\section{Оценивание параметров каналов связи с отражателями в ближнем геометрическом поле}\label{ch:ch2/sec2}

Алгоритм TeNFiL и анализ его эффективности (диссертация Евгения), граница Крамера-Рао (диссертация Лианы), доработки.

\section{Анализ потенциальных характеристик оценивания параметров каналов связи при наличии негауссовских аддитивных помех}\label{ch:ch2/sec3}

Граница Крамера-Рао для общего случая аддитивной помехи с распределением, заданным произвольной смесью Гауссовских распределений.

\begin{comment}
\section{Одиночное изображение}\label{sec:ch2/sec1}

\begin{figure}[ht]
  \centerfloat{
    \includegraphics[scale=0.27]{latex}
  }
  \caption{TeX.}\label{fig:latex}
\end{figure}

Для выравнивания изображения по-центру используется команда \verb+\centerfloat+, которая является во
многом улучшенной версией встроенной команды \verb+\centering+.

\section{Длинное название параграфа, в котором мы узнаём как сделать две картинки с~общим номером и названием}\label{sec:ch2/sect2}

А это две картинки под общим номером и названием:
\begin{figure}[ht]
  \begin{minipage}[b][][b]{0.49\linewidth}\centering
    \includegraphics[width=0.5\linewidth]{knuth1} \\ а)
  \end{minipage}
  \hfill
  \begin{minipage}[b][][b]{0.49\linewidth}\centering
    \includegraphics[width=0.5\linewidth]{knuth2} \\ б)
  \end{minipage}
  \caption{Очень длинная подпись к изображению,
      на котором представлены две фотографии Дональда Кнута}
  \label{fig:knuth}
\end{figure}

Те~же~две картинки под~общим номером и~названием,
но с автоматизированной нумерацией подрисунков:
\begin{figure}[ht]
    \centerfloat{
        \hfill
        \subcaptionbox[List-of-Figures entry]{Первый подрисунок\label{fig:knuth_2-1}}{%
            \includegraphics[width=0.25\linewidth]{knuth1}}
        \hfill
        \subcaptionbox{\label{fig:knuth_2-2}}{%
            \includegraphics[width=0.25\linewidth]{knuth2}}
        \hfill
        \subcaptionbox{Третий подрисунок, подпись к которому
        не~помещается на~одной строке}{%
            \includegraphics[width=0.3\linewidth]{example-image-c}}
        \hfill
    }
    \legend{Подрисуночный текст, описывающий обозначения, например. Согласно
    ГОСТ 2.105, пункт 4.3.1, располагается перед наименованием рисунка.}
    \caption[Этот текст попадает в названия рисунков в списке рисунков]{Очень
    длинная подпись к второму изображению, на~котором представлены две
    фотографии Дональда Кнута}\label{fig:knuth_2}
\end{figure}

На рисунке~\ref{fig:knuth_2-1} показан Дональд Кнут без головного убора.
На рисунке~\ref{fig:knuth_2}\subcaptionref*{fig:knuth_2-2}
показан Дональд Кнут в головном уборе.

Возможно вставлять векторные картинки, рассчитываемые \LaTeX\ <<на~лету>>
с~их~предварительной компиляцией. Надписи в таких рисунках будут выполнены
тем же~шрифтом, который указан для документа в целом.
На~рисунке~\ref{fig:tikz_example} на~странице~\pageref{fig:tikz_example}
представлен пример схемы, рассчитываемой пакетом \verb|tikz| <<на~лету>>.
Для ускорения компиляции, подобные рисунки могут быть <<кешированы>>, что
определяется настройками в~\verb|common/setup.tex|.
Причём имя предкомпилированного
файла и~папка расположения таких файлов могут быть отдельно заданы,
что удобно, если не~для подготовки диссертации,
то~для подготовки научных публикаций.
\begin{figure}[ht]
    \centerfloat{
        \ifdefmacro{\tikzsetnextfilename}{\tikzsetnextfilename{tikz_example_compiled}}{}% присваиваемое предкомпилированному pdf имя файла (не обязательно)
        \input{Dissertation/images/tikz_scheme.tikz}

    }
    \legend{}
    \caption[Пример \texttt{tikz} схемы]{Пример рисунка, рассчитываемого
        \texttt{tikz}, который может быть предкомпилирован}\label{fig:tikz_example}
\end{figure}

Множество программ имеют либо встроенную возможность экспортировать векторную
графику кодом \verb|tikz|, либо соответствующий пакет расширения.
Например, в GeoGebra есть встроенный экспорт,
для Inkscape есть пакет svg2tikz,
для Python есть пакет matplotlib2tikz,
для R есть пакет tikzdevice.

\section{Пример вёрстки списков}\label{sec:ch2/sec3}

\noindent Нумерованный список:
\begin{enumerate}
  \item Первый пункт.
  \item Второй пункт.
  \item Третий пункт.
\end{enumerate}

\noindent Маркированный список:
\begin{itemize}
  \item Первый пункт.
  \item Второй пункт.
  \item Третий пункт.
\end{itemize}

\noindent Вложенные списки:
\begin{itemize}
  \item Имеется маркированный список.
  \begin{enumerate}
    \item В нём лежит нумерованный список,
    \item в котором
    \begin{itemize}
      \item лежит ещё один маркированный список.
    \end{itemize}
  \end{enumerate}
\end{itemize}

\noindent Нумерованные вложенные списки:
\begin{enumerate}
  \item Первый пункт.
  \item Второй пункт.
  \item Вообще, по ГОСТ 2.105 первый уровень нумерации
  (при необходимости ссылки в тексте документа на одно из перечислений)
  идёт буквами русского или латинского алфавитов,
  а второй "--- цифрами со~скобками.
  Здесь отходим от ГОСТ.
    \begin{enumerate}
      \item в нём лежит нумерованный список,
      \item в котором
        \begin{enumerate}
          \item ещё один нумерованный список,
          \item третий уровень нумерации не нормирован ГОСТ 2.105;
          \item обращаем внимание на строчность букв,
          \item в этом списке
          \begin{itemize}
            \item лежит ещё один маркированный список.
          \end{itemize}
        \end{enumerate}

    \end{enumerate}

  \item Четвёртый пункт.
\end{enumerate}

\section{Традиции русского набора}

Много полезных советов приведено в материале
<<\href{http://www.dropbox.com/s/x4hajy4pkw3wdql/wholesome-typesetting.pdf?dl=1\&pv=1}{Краткий курс благородного набора}>> (автор А.\:В.~Костырка).
Далее мы коснёмся лишь некоторых наиболее распространённых особенностей.

\subsection{Пробелы}

В~русском наборе принято:
\begin{itemize}
    \item единицы измерения, знак процента отделять пробелами от~числа:
        10~кВт, 15~\% (согласно ГОСТ 8.417, раздел 8);
    \item \(\tg 20\text{\textdegree}\), но: 20~{\textdegree}C
        (согласно ГОСТ 8.417, раздел 8);
    \item знак номера, параграфа отделять от~числа: №~5, \S~8;
    \item стандартные сокращения: т.\:е., и~т.\:д., и~т.\:п.;
    \item неразрывные пробелы в~предложениях.
\end{itemize}

\subsection{Математические знаки и символы}

Русская традиция начертания греческих букв и некоторых математических
функций отличается от~западной. Это исправляется серией
\verb|\renewcommand|.
\begin{itemize}
%Все \original... команды заранее, ради этого примера, определены в Dissertation\userstyles.tex
    \item[До:] \( \originalepsilon \originalge \originalphi\),
    \(\originalphi \originalleq \originalepsilon\),
    \(\originalkappa \in \originalemptyset\),
    \(\originaltan\),
    \(\originalcot\),
    \(\originalcsc\).
    \item[После:] \( \epsilon \ge \phi\),
    \(\phi \leq \epsilon\),
    \(\kappa \in \emptyset\),
    \(\tan\),
    \(\cot\),
    \(\csc\).
\end{itemize}

Кроме того, принято набирать греческие буквы вертикальными, что
решается подключением пакета \verb|upgreek| (см. закомментированный
блок в~\verb|userpackages.tex|) и~аналогичным переопределением в
преамбуле (см.~закомментированный блок в~\verb|userstyles.tex|). В
этом шаблоне такие переопределения уже включены.

Знаки математических операций принято переносить. Пример переноса
в~формуле~\eqref{eq:equation3}.

\subsection{Кавычки}
В английском языке приняты одинарные и двойные кавычки в~виде ‘...’ и~“...”.
В России приняты французские («...») и~немецкие („...“) кавычки (они называются
«ёлочки» и~«лапки», соответственно). ,,Лапки`` обычно используются внутри
<<ёлочек>>, например, <<... наш гордый ,,Варяг``...>>.

Французкие левые и правые кавычки набираются
как лигатуры \verb|<<| и~\verb|>>|, а~немецкие левые
и правые кавычки набираются как лигатуры \verb|,,| и~\verb|‘‘| (\verb|``|).

Вместо лигатур или команд с~активным символом "\ можно использовать команды
\verb|\glqq| и \verb|\grqq| для набора немецких кавычек и команды \verb|\flqq|
и~\verb|\frqq| для набора французских кавычек. Они определены в пакете
\verb|babel|.

\subsection{Тире}
%  babel+pdflatex по умолчанию, в polyglossia надо включать опцией (и перекомпилировать с удалением временных файлов)
Команда \verb|"---| используется для печати тире в тексте. Оно несколько короче
английского длинного тире. Кроме того, команда задаёт небольшую жёсткую отбивку
от слова, стоящего перед тире. При этом, само тире не~отрывается от~слова.
После тире следует такая же отбивка от текста, как и~перед тире. При наборе
текста между словом и командой, за которым она следует, должен стоять пробел.

В составных словах, таких, как <<Закон Менделеева"--~Клапейрона>>, для печати
тире надо использовать команду \verb|"--~|. Она ставит более короткое,
по~сравнению с~английским, тире и позволяет делать переносы во втором слове.
При~наборе текста команда \verb|"--~| не отделяется пробелом от слова,
за~которым она следует (\verb|Менделеева"--~|). Следующее за командой слово
может быть  отделено от~неё пробелом или перенесено на другую строку.

Если прямая речь начинается с~абзаца, то перед началом её печатается тире
командой \verb|"--*|. Она печатает русское тире и жёсткую отбивку нужной
величины перед текстом.

\subsection{Дефисы и переносы слов}
%  babel+pdflatex по умолчанию, в polyglossia надо включать опцией (и перекомпилировать с удалением временных файлов)
Для печати дефиса в~составных словах введены две команды. Команда~\verb|"~|
печатает дефис и~запрещает делать переносы в~самих словах, а~команда \verb|"=|
печатает дефис, оставляя \TeX ’у право делать переносы в~самих словах.

В отличие от команды \verb|\-|, команда \verb|"-| задаёт место в~слове, где
можно делать перенос, не~запрещая переносы и~в~других местах слова.

Команда \verb|""| задаёт место в~слове, где можно делать перенос, причём дефис
при~переносе в~этом месте не~ставится.

Команда \verb|",| вставляет небольшой пробел после инициалов с~правом переноса
в~фамилии.

\section{Текст из панграмм и формул}

\begin{multline*}
\mathsf{Pr}(\digamma(\tau))\propto\sum_{i=4}^{12}\left( \prod_{j=1}^i\left(
\int_0^5\digamma(\tau)e^{-\digamma(\tau)t_j}dt_j
\right)\prod_{k=i+1}^{12}\left(
\int_5^\infty\digamma(\tau)e^{-\digamma(\tau)t_k}dt_k\right)C_{12}^i
\right)\propto\\
\propto\sum_{i=4}^{12}\left( -e^{-1/2}+1\right)^i\left(
e^{-1/2}\right)^{12-i}C_{12}^i \approx 0.7605,\quad
\forall\tau\neq\overline{\tau}
\end{multline*}

%Большая фигурная скобка только справа
\[\left. %ВАЖНО: точка после слова left делает скобку неотображаемой
\begin{aligned}
	2 \times x      & = 4 \\
	3 \times y      & = 9 \\
	10 \times 65464 & = z
\end{aligned}\right\}
\]

\end{comment}     % Глава 2
\chapter{Практические приложения предложенных методов}\label{ch:ch3}

\section{Применение в интегрированных системах связи и локации}\label{ch:ch3/sec1}

Концепции интегрированных систем. Параметры каналов связи, фактически, имеют двоякую применимость и могут быть интересны сами по себе для целей локализации и т. д.
В этой части нет собственных статей, пока что можно только описать как предложенные методы могут применяться в таких системах. 

\section{Применение в задаче локализации акустических источников}\label{ch:ch3/sec2}

Акустические сигналы широкополосны по своей природе, а также нужно проработать возможность наличия источников в ближнем геометрическом поле. Также возможно нужно усложнить модель с учетом рассеянного отражения (diffuse scattering). Благо, программа для испытаний/измерений уже есть, нужен больше анализ и правильная обработка.

\begin{comment}
\section{Таблица обыкновенная}\label{sec:ch3/sect1}

Так размещается таблица:

\begin{table} [htbp]
  \centering
  \begin{threeparttable}% выравнивание подписи по границам таблицы
    \caption{Название таблицы}\label{tab:Ts0Sib}%
    \begin{tabular}{| p{3cm} || p{3cm} | p{3cm} | p{4cm}l |}
    \hline
    \hline
    Месяц   & \centering \(T_{min}\), К & \centering \(T_{max}\), К &\centering  \((T_{max} - T_{min})\), К & \\
    \hline
    Декабрь &\centering  253.575   &\centering  257.778    &\centering      4.203  &   \\
    Январь  &\centering  262.431   &\centering  263.214    &\centering      0.783  &   \\
    Февраль &\centering  261.184   &\centering  260.381    &\centering     \(-\)0.803  &   \\
    \hline
    \hline
    \end{tabular}
  \end{threeparttable}
\end{table}

\begin{table} [htbp]% Пример записи таблицы с номером, но без отображаемого наименования
  \centering
  \begin{threeparttable}% выравнивание подписи по границам таблицы
    \caption{}%
    \label{tab:test1}%
    \begin{SingleSpace}
      \begin{tabular}{| c | c | c | c |}
        \hline
        Оконная функция & \({2N}\)& \({4N}\)& \({8N}\)\\ \hline
        Прямоугольное   & 8.72  & 8.77  & 8.77  \\ \hline
        Ханна           & 7.96  & 7.93  & 7.93  \\ \hline
        Хэмминга        & 8.72  & 8.77  & 8.77  \\ \hline
        Блэкмана        & 8.72  & 8.77  & 8.77  \\ \hline
      \end{tabular}%
    \end{SingleSpace}
  \end{threeparttable}
\end{table}

Таблица~\ref{tab:test2} "--- пример таблицы, оформленной в~классическом книжном
варианте или~очень близко к~нему. \mbox{ГОСТу} по~сути не~противоречит. Можно
ещё~улучшить представление, с~помощью пакета \verb|siunitx| или~подобного.

\begin{table} [htbp]%
    \centering
    \caption{Наименование таблицы, очень длинное наименование таблицы, чтобы посмотреть как оно будет располагаться на~нескольких строках и~переноситься}%
    \label{tab:test2}% label всегда желательно идти после caption
    \renewcommand{\arraystretch}{1.5}%% Увеличение расстояния между рядами, для улучшения восприятия.
    \begin{SingleSpace}
        \begin{tabular}{@{}@{\extracolsep{20pt}}llll@{}} %Вертикальные полосы не используются принципиально, как и лишние горизонтальные (допускается по ГОСТ 2.105 пункт 4.4.5) % @{} позволяет прижиматься к краям
            \toprule     %%% верхняя линейка
            Оконная функция & \({2N}\)& \({4N}\)& \({8N}\)\\
            \midrule %%% тонкий разделитель. Отделяет названия столбцов. Обязателен по ГОСТ 2.105 пункт 4.4.5
            Прямоугольное   & 8.72  & 8.77  & 8.77  \\
            Ханна           & 7.96  & 7.93  & 7.93  \\
            Хэмминга        & 8.72  & 8.77  & 8.77  \\
            Блэкмана        & 8.72  & 8.77  & 8.77  \\
            \bottomrule %%% нижняя линейка
        \end{tabular}%
    \end{SingleSpace}
\end{table}

\section{Таблица с многострочными ячейками и примечанием}

В таблице~\ref{tab:makecell} приведён пример использования команды
\verb+\multicolumn+ для объединения горизонтальных ячеек таблицы,
и команд пакета \textit{makecell} для добавления разрыва строки внутри ячеек.
При форматировании таблицы~\ref{tab:makecell} использован стиль подписей \verb+split+.
Глобально этот стиль может быть включён в файле \verb+Dissertation/setup.tex+ для диссертации и в
файле \verb+Synopsis/setup.tex+ для автореферата.
Однако такое оформление не соответствует ГОСТ.

\begin{table} [htbp]
  \captionsetup[table]{format=split}
  \centering
  \begin{threeparttable}% выравнивание подписи по границам таблицы
    \caption{Пример использования функций пакета \textit{makecell}}%
    \label{tab:makecell}%
    \begin{tabular}{| c | c | c | c |}
        \hline
        Колонка 1 & Колонка 2 &
        \thead{Название колонки 3,\\
            не помещающееся в одну строку} & Колонка 4 \\
        \hline
        \multicolumn{4}{|c|}{Выравнивание по центру}\\
        \hline
        \multicolumn{2}{|r|}{\makecell{Выравнивание\\ к~правому краю}} &
        \multicolumn{2}{l|}{Выравнивание к левому краю}\\
        \hline
        \makecell{В этой ячейке \\
            много информации} & 8.72 & 8.55 & 8.44\\
        \cline{3-4}
        А в этой мало         & 8.22 & \multicolumn{2}{c|}{5}\\
        \hline
    \end{tabular}%
  \end{threeparttable}
\end{table}

Таблицы~\ref{tab:test3} и~\ref{tab:test4} "--- пример реализации расположения
примечания в~соответствии с ГОСТ 2.105. Каждый вариант со своими достоинствами
и~недостатками. Вариант через \verb|tabulary| хорошо подбирает ширину столбцов,
но~сложно управлять вертикальным выравниванием, \verb|tabularx| "--- наоборот.
\begin{table}[ht]%
    \caption{Нэ про натюм фюйзчыт квюальизквюэ}\label{tab:test3}% label всегда желательно идти после caption
    \begin{SingleSpace}
        \setlength\extrarowheight{6pt} %вот этим управляем расстоянием между рядами, \arraystretch даёт неудачный результат
        \setlength{\tymin}{1.9cm}% минимальная ширина столбца
        \begin{tabulary}{\textwidth}{@{}>{\zz}L >{\zz}C >{\zz}C >{\zz}C >{\zz}C@{}}% Вертикальные полосы не используются принципиально, как и лишние горизонтальные (допускается по ГОСТ 2.105 пункт 4.4.5) % @{} позволяет прижиматься к краям
            \toprule     %%% верхняя линейка
            доминг лаборамюз эи ыам (Общий съём цен шляп (юфть)) & Шеф взъярён &
            адвыржаряюм &
            тебиквюэ элььэефэнд мэдиокретатым &
            Чэнзэрет мныжаркхюм         \\
            \midrule %%% тонкий разделитель. Отделяет названия столбцов. Обязателен по ГОСТ 2.105 пункт 4.4.5
            Эй, жлоб! Где туз? Прячь юных съёмщиц в~шкаф Плюш изъят. Бьём чуждый цен хвощ! &
            \({\approx}\) &
            \({\approx}\) &
            \({\approx}\) &
            \( + \) \\
            Эх, чужак! Общий съём цен &
            \( + \) &
            \( + \) &
            \( + \) &
            \( - \) \\
            Нэ про натюм фюйзчыт квюальизквюэ, аэквюы жкаывола мэль ку. Ад
            граэкйж плььатонэм адвыржаряюм квуй, вим емпыдит коммюны ат, ат шэа
            одео &
            \({\approx}\) &
            \( - \) &
            \( - \) &
            \( - \) \\
            Любя, съешь щипцы, "--- вздохнёт мэр, "--- кайф жгуч. &
            \( - \) &
            \( + \) &
            \( + \) &
            \({\approx}\) \\
            Нэ про натюм фюйзчыт квюальизквюэ, аэквюы жкаывола мэль ку. Ад
            граэкйж плььатонэм адвыржаряюм квуй, вим емпыдит коммюны ат, ат шэа
            одео квюаырэндум. Вёртюты ажжынтиор эффикеэнди эож нэ. &
            \( + \) &
            \( - \) &
            \({\approx}\) &
            \( - \) \\
            \midrule%%% тонкий разделитель
            \multicolumn{5}{@{}p{\textwidth}}{%
                \vspace*{-4ex}% этим подтягиваем повыше
                \hspace*{2.5em}% абзацный отступ - требование ГОСТ 2.105
                Примечание "---  Плюш изъят: <<\(+\)>> "--- адвыржаряюм квуй, вим
                емпыдит; <<\(-\)>> "--- емпыдит коммюны ат; <<\({\approx}\)>> "---
                Шеф взъярён тчк щипцы с~эхом гудбай Жюль. Эй, жлоб! Где туз?
                Прячь юных съёмщиц в~шкаф. Экс-граф?
            }
            \\
            \bottomrule %%% нижняя линейка
        \end{tabulary}%
    \end{SingleSpace}
\end{table}

Если таблица~\ref{tab:test3} не помещается на той же странице, всё
её~содержимое переносится на~следующую, ближайшую, а~этот текст идёт перед ней.
\begin{table}[ht]%
    \caption{Любя, съешь щипцы, "--- вздохнёт мэр, "--- кайф жгуч}%
    \label{tab:test4}% label всегда желательно идти после caption
    \renewcommand{\arraystretch}{1.6}%% Увеличение расстояния между рядами, для улучшения восприятия.
    \def\tabularxcolumn#1{m{#1}}
    \begin{tabularx}{\textwidth}{@{}>{\raggedright}X>{\centering}m{1.9cm} >{\centering}m{1.9cm} >{\centering}m{1.9cm} >{\centering\arraybackslash}m{1.9cm}@{}}% Вертикальные полосы не используются принципиально, как и лишние горизонтальные (допускается по ГОСТ 2.105 пункт 4.4.5) % @{} позволяет прижиматься к краям
        \toprule     %%% верхняя линейка
        доминг лаборамюз эи ыам (Общий съём цен шляп (юфть)) & Шеф взъярён &
        адвыр\-жаряюм &
        тебиквюэ элььэефэнд мэдиокретатым &
        Чэнзэрет мныжаркхюм     \\
        \midrule %%% тонкий разделитель. Отделяет названия столбцов. Обязателен по ГОСТ 2.105 пункт 4.4.5
        Эй, жлоб! Где туз? Прячь юных съёмщиц в~шкаф Плюш изъят.
        Бьём чуждый цен хвощ! &
        \({\approx}\) &
        \({\approx}\) &
        \({\approx}\) &
        \( + \) \\
        Эх, чужак! Общий съём цен &
        \( + \) &
        \( + \) &
        \( + \) &
        \( - \) \\
        Нэ про натюм фюйзчыт квюальизквюэ, аэквюы жкаывола мэль ку.
        Ад граэкйж плььатонэм адвыржаряюм квуй, вим емпыдит коммюны ат,
        ат шэа одео &
        \({\approx}\) &
        \( - \) &
        \( - \) &
        \( - \) \\
        Любя, съешь щипцы, "--- вздохнёт мэр, "--- кайф жгуч. &
        \( - \) &
        \( + \) &
        \( + \) &
        \({\approx}\) \\
        Нэ про натюм фюйзчыт квюальизквюэ, аэквюы жкаывола мэль ку. Ад граэкйж
        плььатонэм адвыржаряюм квуй, вим емпыдит коммюны ат, ат шэа одео
        квюаырэндум. Вёртюты ажжынтиор эффикеэнди эож нэ. &
        \( + \) &
        \( - \) &
        \({\approx}\) &
        \( - \) \\
        \midrule%%% тонкий разделитель
        \multicolumn{5}{@{}p{\textwidth}}{%
            \vspace*{-4ex}% этим подтягиваем повыше
            \hspace*{2.5em}% абзацный отступ - требование ГОСТ 2.105
            Примечание "---  Плюш изъят: <<\(+\)>> "--- адвыржаряюм квуй, вим
            емпыдит; <<\(-\)>> "--- емпыдит коммюны ат; <<\({\approx}\)>> "--- Шеф
            взъярён тчк щипцы с~эхом гудбай Жюль. Эй, жлоб! Где туз? Прячь юных
            съёмщиц в~шкаф. Экс-граф?
        }
        \\
        \bottomrule %%% нижняя линейка
    \end{tabularx}%
\end{table}

\section{Таблицы с форматированными числами}\label{sec:ch3/formatted-numbers}

В таблицах~\refs{tab:S:parse,tab:S:align} представлены примеры использования опции
форматирования чисел \texttt{S}, предоставляемой пакетом \texttt{siunitx}.

\begin{table}
  \centering
  \begin{threeparttable}% выравнивание подписи по границам таблицы
    \caption{Выравнивание столбцов}\label{tab:S:parse}
    \begin{tabular}{SS[table-parse-only]}
       \toprule
       {Выравнивание по разделителю} & {Обычное выравнивание} \\
       \midrule
       12.345                        & 12.345                 \\
       6,78                          & 6,78                   \\
       -88.8(9)                      & -88.8(9)               \\
       4.5e3                         & 4.5e3                  \\
       \bottomrule
    \end{tabular}
  \end{threeparttable}
\end{table}

\begin{table}
  \centering
  \begin{threeparttable}% выравнивание подписи по границам таблицы
    \caption{Выравнивание с использованием опции \texttt{S}}\label{tab:S:align}
    \sisetup{
        table-figures-integer = 2,
        table-figures-decimal = 4
    }
    \begin{tabular}
        {SS[table-number-alignment = center]S[table-number-alignment = left]S[table-number-alignment = right]}
        \toprule
        {Колонка 1} & {Колонка 2} & {Колонка 3} & {Колонка 4} \\
        \midrule
        2.3456      & 2.3456      & 2.3456      & 2.3456      \\
        34.2345     & 34.2345     & 34.2345     & 34.2345     \\
        56.7835     & 56.7835     & 56.7835     & 56.7835     \\
        90.473      & 90.473      & 90.473      & 90.473      \\
        \bottomrule
    \end{tabular}
  \end{threeparttable}
\end{table}

\section{Параграф "--- два}\label{sec:ch3/sect2}

Некоторый текст.

\section{Параграф с подпараграфами}\label{sec:ch3/sect3}

\subsection{Подпараграф "--- один}\label{subsec:ch3/sect3/sub1}

Некоторый текст.

\subsection{Подпараграф "--- два}\label{subsec:ch3/sect3/sub2}

Некоторый текст.

\clearpage

\end{comment}     % Глава 3

\clearpage
\chapter{Заключение}
Основные результаты работы заключаются в следующем:
%% Согласно ГОСТ Р 7.0.11-2011:
%% 5.3.3 В заключении диссертации излагают итоги выполненного исследования, рекомендации, перспективы дальнейшей разработки темы.
%% 9.2.3 В заключении автореферата диссертации излагают итоги данного исследования, рекомендации и перспективы дальнейшей разработки темы.

\begin{itemize}
\item предложен метод предварительной обработки многомерных сигналов в широкополосных системах, позволяющий применять алгоритмы оценивания параметров каналов связи, разработанные для узкополосных систем. Предложенный метод предварительной обработки многомерных сигналов в широкополосных системах существенно повышает точность алгоритмов оценки параметров каналов связи, разработанных для узкополосных систем. Улучшение точности оценивания увеличивается при увеличении относительной полосы частот рассматриваемой широкополосной системы. Например, показана эффективность данного метода в широкополосных системах с относительной полосой частот, превышающей 10\%; 
\item разработан новый алгоритм оценивания параметров канала связи с отражателями в ближнем геометрическом поле с использованием сферической модели фронта волны. С помощью компьютерного моделирования показано, что разработанный алгоритм оценивания параметров канала связи с отражателями в ближнем геометрическом поле более эффективен чем существующие алгоритмы;
\item впервые исследована граница Крамера-Рао для задач оценки параметров каналов связи с негауссовым распределением аддитивной помехи, заданным смесью нормальных распределений с ненулевыми средними значениями компонент. Показано, что использование более сложных моделей аддитивных помех, в частности с негауссовским распределением аддитивной помехи, выраженным смесью нормальных распределений, потенциально позволяет увеличить точность работы алгоритмов оценивания параметров каналов связи;
\item разработана программная реализация предложенных алгоритмов, а также компьютерные модели, имитирующие работу инфокоммуникационных систем с предложенными алгоритмами и оценивающие эффективность их работы.
\end{itemize}

\begin{comment}
Результаты математического моделирования показали практическую применимость метода вычисления нижней границы Крамера-Рао. Используя данный метод граница может быть вычислена при произвольном распределении аддитивной помехи. Результаты моделирования также показали, что граница Крамера-Рао ниже для сложных распределений аддитивных помех. Следовательно, алгоритмы на основе сложных моделей аддитивных помех содержат потенциал к улучшению качества оценивания искомых параметров, и, как следствие, к улучшению качественных показателей функционирования систем радиолокации и телекоммуникаций.
\end{comment}

\clearpage
\addcontentsline{toc}{chapter}{Публикации автора по теме диссертации}
\ifdefmacro{\microtypesetup}{\microtypesetup{protrusion=false}}{} % не рекомендуется применять пакет микротипографики к автоматически генерируемому списку литературы
\urlstyle{rm}                               % ссылки URL обычным шрифтом
\ifnumequal{\value{bibliosel}}{0}{% Встроенная реализация с загрузкой файла через движок bibtex8
  \renewcommand{\bibname}{\large \bibtitleauthor}
  \nocite{*}
  \insertbiblioauthor           % Подключаем Bib-базы
  %\insertbiblioexternal   % !!! bibtex не умеет работать с несколькими библиографиями !!!
}{% Реализация пакетом biblatex через движок biber
  % Цитирования.
  %  * Порядок перечисления определяет порядок в библиографии (только внутри подраздела, если `\insertbiblioauthorgrouped`).
  %  * Если не соблюдать порядок "как для \printbibliography", нумерация в `\insertbiblioauthor` будет кривой.
  %  * Если цитировать каждый источник отдельной командой --- найти некоторые ошибки будет проще.
  %
  %% authorvak
  % \nocite{*}
  \nocite{vakbib1}%
  \nocite{vakbib2}%
  %
  %% authorscopus
  \nocite{Zhang2017}%
  \nocite{Podkurkov2017}% problem???
  %
  %% authorconf
  \nocite{Zhang2016}%
  \nocite{ptitt18}%
  \nocite{Podkurkov2018}%
  %% authorwos
  \nocite{Podkurkov2019}%
  %% authorother
  \nocite{otherbib1}%
  %
  \ifnumgreater{\value{usefootcite}}{0}{
    \begin{refcontext}[labelprefix={}]
      \ifnum \value{bibgrouped}>0
        \insertbiblioauthorgrouped    % Вывод всех работ автора, сгруппированных по источникам
      \else
        \insertbiblioauthor      % Вывод всех работ автора
      \fi
    \end{refcontext}
  }{
  \ifnum \value{citeexternal}>0
    \begin{refcontext}[labelprefix=A]
      \ifnum \value{bibgrouped}>0
        \insertbiblioauthorgrouped    % Вывод всех работ автора, сгруппированных по источникам
      \else
        \insertbiblioauthor      % Вывод всех работ автора
      \fi
    \end{refcontext}
  \else
    \ifnum \value{bibgrouped}>0
      \insertbiblioauthorgrouped    % Вывод всех работ автора, сгруппированных по источникам
    \else
      \insertbiblioauthor      % Вывод всех работ автора
    \fi
  \fi
  %  \insertbiblioauthorimportant  % Вывод наиболее значимых работ автора (определяется в файле characteristic во второй section)
  \begin{refcontext}[labelprefix={}]    \insertbiblioexternal            % Вывод списка литературы, на которую ссылались в тексте автореферата
  \end{refcontext}
  }
}
\ifdefmacro{\microtypesetup}{\microtypesetup{protrusion=true}}{}
\urlstyle{tt}                               % возвращаем установки шрифта ссылок URL
